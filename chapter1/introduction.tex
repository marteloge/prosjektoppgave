\chapter{Introduction}

  \section{Background and Motivation}

  In todays society, people tends to spend a lot of time on their mobile devices. Mobile devices are not just a communication, but also an important tool for every day tasks like doing our work, reading mail, pay our bills, and keeping up with our social life. Our whole life is contained in one device. When such a small device is so central in our every day life, it makes it vulnerable.

  Passwords are human-chosen secrets that are connected to you as a person. When the password are created you might create a password that is a association to something you know or recognize; passwords are more than just words and numbers. Because of the shortcomings with text-based passwords \cite{UnixPasswords}, there is an increased interest in graphical passwords. The interest in graphical passwords started by the assumption that pictures are easier to remember and more secure than words and numbers \cite{DeAngeli}. Google's Android platform released the functionality for the Android Unlock Pattern in 2008, that is a security mechanism for locking the mobile phone. Since its release, there have been a lot of discussion of its security, but few researchers have done a scientific research on the Android Unlock Pattern. The problem is not just the theoretical password space, but the password space in practice. 

  The motivation for this thesis started by observing the shortcomings with graphical passwords. Password reuse is one of the known password habits among users because the human limitation to remembering text-based password. Some users also make simple or meaningful password that are easier to remember, making their passwords vulnerable to attacks.Graphical passwords looks like a promising alternative to text-based passwords, as it supports users to remember and make more complex passwords, offering better usability and higher security. As mobile devices plays an important role in our everyday life, it makes it interesting target device for passwords. Security on mobile devices have changed during the past years. The history of locking mechanisms was often a solution solely to prevent accidental use, while current mobile phones require protection in order to secure the potentially vast amount of private data that we keep on our smart phones. The situation of our rapidly use of mobile phones, as well as it well suited platform for graphical password, makes authentication on mobile devices an interesting field of study.

  In 2013 a research group conducted the first large-scale user study on the Android Unlock Patterns \cite{Uellenbeck}, that is a security mechanism on featured in the Android smart phones. The outcome of the research was a analysis of 2900 collected Android Unlock Patterns. They found a lot of bias in the pattern making process concluding that the scheme are less secure than its stated theoretical password space. 

  This research aims to take the analysis of peoples choice in Android Unlock Patterns a step further by including the human properties that may impact the user choice in graphical passwords. This thesis is the first phase of my research, and will continued in my master thesis in the following spring. 

  \section{Research Question}
    
    This thesis is the first phase of my research, and will be a supporting work for my master thesis. 

    {\bf $RQ1$: What is the status of current research on graphical passwords?} \\

    {\bf $RQ2$: What human properties may affect our choice of graphical passwords on mobile devices?}

  \section{Deliverables}

    The deliverables from this thesis is a literature review on graphical passwords to get an overview of published research, and provide a conceptual framework for the rest of the thesis. A literature review is providing the information needed to decide on a main research hypothesis for my master thesis, and the aim is to fill the gap in research on graphical passwords on mobile devices.

    As well as a literature review, there will be delivered a research design that will be used further in my master thesis. The research design is mainly a detailed description of the research strategy and data collection method. Research on passwords is not easy to conduct because of the nature of passwords. Passwords should remain a secret for the user, and as we have learned, we should not share our password due to security concerns. Research on text-based passwords is often based on leaked password on the web. When analyzing graphical passwords on mobile devices, there is no such data source available. 
    The research design from this research can provide insight into a new way of solving the problem of collecting user chosen passwords from mobile devices. This can provide knowledge for future research on graphical passwords on mobile devices.

  \section{Scope}

    In this thesis I have conducted a literature review on graphical passwords. 

    \todo{Beskrive fremgangsmåte og forskningsprosess}
    \todo{Hvilke ord har jeg søkt på og hvilke evalueringskriterier har jeg brukt?}

  \section{Limitations}

    The aim of this thesis is to design a research within a problem domain of graphical passwords that not have been answered yet. This have been solved by conducting a ad-hoc literature review. Due to the choice of ad-hoc instead of a structured literature review may impact the results in the literature that is found. 

    \todo{Beskrive at måten å gjennomføre litterature review kan påvirke resultat}

  \section{Thesis structure}

    {\bf Chapter 2: Background Theory} presents the relevant background and theory that might improve the understanding of the information in the literature review. 

    {\bf Chapter 3: Literature Review} provides an overview of the current research on graphical passwords. 

    {\bf Chapter 4: Research Design} present the chosen research strategy and data collection method that will be used in my master thesis. 

    {\bf Chapter 5: Discussion and Further Work} will be a discussion of the results from this thesis, and give a brief summary of the results. Further work will give a description of what will be conducted in the next phase of this research, my master thesis.  





