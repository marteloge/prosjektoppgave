\chapter{Introduction}

\section{Background and Motivation}
    
  % In todays society we're addicted to our mobile devices in our every day life. Mobile devices are not just a communication tool for calling and texting, but also an important tool for every day tasks like doing our work, reading mail, pay our bills and keeping up with our social life. Our whole life is contained in one device! When such a small device is so imortant, it makes it vurnerable. How do we secure it?

  % %Information security have become a critical concern from a business point of view

  % There are many different security tips for mobile devices, but the most imortant of them is locking your device with a password. There are many different password schemes, but the most commonly used password schemes are PINs, passphrases and graphical passwords.

  % The interest in graphical passwords started by the assumption that pictures are easier to remember and more secure than words and numbers. Google's Android platform released the  functionality for Unlock Unlock Patterns in 2008. The Android Unlock pattern is a graphical password schemes that asks the user to make a pattern on a 3x3 grid by making a patten of connected nodes. Since its relese there have been a lot of discussion of its security, but few researchers have done a scientific reseach on this. The problem is not just the theoretical password space, but the password space in practice.

  % In 2013 a research group conducted the first large-scale user study on Android Unlock Patterns \cite{Uellenbeck}. The outcome of the research was a analysis of 2900 collected Android Unlock Patterns. They found a lot of bias in the pattern making process cocluding that the schemes are less secure than its theoretical security.

\clearpage
\section{Research Questions and Goals}

  The aim of this project is to design an experiment for collecting graphical passwords that further will be used in my master thesis the following spring. In the experiment it will be collected passwords, as well as information about the users creating the passwords. Before designing the experiment this thesis will include a detailed state-of-the-art study on passwords. In order to understand the human factors that impacts how people make their passwords, this thesis will also include a study of the psychological aspects that connects humans and passwords. This are covered in my research questions below:

  \subsection*{Research Questions}

    {\bf RQ1: What is the status of current research on graphical passwords? } \\
    Research are always moving forward with new hypothesises and new results in the field. In order to do a research is imortant to know the relevant work as well as avoid answearing questions that already have been aswered.
    
    {\bf RQ2: What human factors may affect our choice of passwords?} \\
    Passwords are human-chosen secrets that are only connected to you as a person. When the secret are created you might create a password that are a association to something you know or recognice; passwords are more than just words and numbers. It is important to study the bias introduced in the password making process that can be a cause of human factors. Psychology are a field of study that might can give a understanding of how we think and give an explanation of why we make the choices that we do. 
    
    {\bf RQ3: How should an experiment to collect graphical passwords set by varoius user types be designed?} \\
    It is important to consider the biases that can be introduced as a cause of the experiment design. The design have to consider what data that needs to be collected and how the data should be collected. It is also important to consider the diversity of the data in order to be able to get good resluts from the analysis. The result will be a detailed design on the experiment that will be conducted in the following spring. 

\section{Research Method}


\section{Expected Deliverables}
  

\section{Thesis structure}





% \item To what extent can graphical elements like colors, shapes, and objects infuence the end-users choice of passwords?
% \item How is features describing the end-user picked, and how do the features relate to end-user's choice of passwords?
% \item Kan grafiske elementer påvirke brukerens valg av passord?
% \item Hvilke kjennetegn ved en person kan gi utslag på valg av passord?
% \item Hvordan skal innsamling av passord skjer for å ivareta datasamlingens pålitelighet?

% \item To what extent can we relate existing research on users choice of alphanumeric passwords to users choice of Android unlock patterns? 
% \item How should passwords from end-users be collected in order to preserve the reliability of the data? 
% \item In order to analyse collected Android unlock patterns, how much data is needed to be collected, and how should the diversity of the data look like?
% \item What kind of data should be included in the data model?
% \item How should the data model be designed in order to cover relevant data for the analysis of the collected data?  
% \item What are the status on research on mobile security?
% \item How should passwords from end-users be collected in order to preserve the reliability of the data? 
