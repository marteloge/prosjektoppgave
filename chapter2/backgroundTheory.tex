\chapter{Background Theory}

  \section{Authentication}

  Authentication is the process of verifying whether a particular individual or a device should be granted access to a system or application running on a device \cite{IPAS}, e.g. verifying that you are the person that you claim to be.

  There are various authentication schemes described in the literature, but they can all be grouped by the following characteristics \cite{IPAS}:

    \begin{itemize}
      \item Who you are
      \item Something you have, and
      \item Something you know
    \end{itemize}

    \subsection{Biometric Authentication}
    Biometric authentication have the characteristics of ``who you are''. Biometric authentication refers to verify a persons identity based on physical or behavioral characteristics of an individual \cite{biometrics, biometrics2}. Biometric authentication are different than other authentications schemes because:

      \begin{itemize}
        \item the biometric password cannot be lost nor forgotten
        \item biometric passwords tends to be difficult to copy, share and distribute, and 
        \item the person being authenticated needs to be present in the authentication process
      \end{itemize} 

     Physiological biometrics uses the physiological characteristics of an individual in the authentication process. The verification uses unique characteristics of a human, e.g. physical parts of the body that are unique like fingerprints, face, iris, hand and finger geometry, and DNA. Behavioral biometrics analyze how a person performs different activities, e.g. applies pattern recognition techniques for activities like keyboard writing, talking and handwriting.

    \subsection{Token-based Authentication}
    In a token-based authentication process the user uses ``something you have'' that is often stored on a physical device. Token-based authentication is often combined with a ``something you know'', making a strong authentication by combining two or more authentication characteristics. In many banking systems you have to use more than just something you know, but also ``something you have'', like a one-time password to pass the authentication process. The one time password is a password that is randomly generated and sent to a physical device, or over an SMS to your mobile phone. This is an extra layer of security, because even if someone steals or know your password, they still cant get access to your banking account because they also would need your security token.

    \subsection{Knowledge-based Authentication}
    ``Something you know'' is often used in the classical login situation where the user have to remember a username/password to get access to a system or device. Some of the commonly used passwords schemes are PIN's, alphanumeric passwords and graphical passwords that all are passwords with the characteristic of ``something you know''.

      {\bf PIN's} Personal identification number is a numeric passwords. The PIN was first introduced in the first ATM in London in 1967 as a efficient way for the banks to authenticate their customers \cite{Bonneau1}.      

      {\bf Alphanumeric passwords}
      The word ``Alphanumeric'' is a composition of the words ``alpha'' (as in alphabet), and ``numeric''(as in numbers). The alphanumeric password may also contain special characters, so in short a alphanumeric password is a mix of all writable characters.

      {\bf Graphical passwords}
      A graphical password have the characteristics of ``something you know'', but instead of using letters and numbers it uses graphical elements as a secret. Graphical passwords was proposed as a alternative to PINs and alphanumeric values because humans tends to remember graphical elements better than letters and numbers. A variety of graphical passwords schemes have been created over the past years. Biddle et al. have collected research of the past decade on graphical password schemes \cite{Biddle}, dividing the schemes into three categories: 

      \begin{itemize}
        \item Recall-based authentication
        \item Recognition-based authentication, and 
        \item Cued-recall authentication.
      \end{itemize}

      Recall-based graphical passwords are often referred to as drawmetric systems \cite{DeAngeli} because the user are are reproducing a secret drawing. The password is normally drawn in a grid or a blank canvas, requiring the user to reproduce the secret password from its memory.

      Recognition-based passwords are often referred to as cognometric systems \cite{DeAngeli} because the user recall a secret drawing, or sequence of drawings, and the reproduces it as the secret password.

      Cued-recall are often referred to locimetric systems \cite{DeAngeli}. With cued-recall authentication typically require the users to remember and target a specific location within and image. This is a version of a recall-based authentication, but helps the user with the recall by showing an image and not just an grid or canvas. It is also different from the recognition-based approach because the user need to identify specific locations in an image as a whole. 

  \section{Key Security Aspects in Authentication}

    In order to be able to evaluate the security of different password schemes, this section will give a brief introduction to key security aspects with knowledge-based authentication, hereafter called passwords.In terms of security, the primary goal of authentication is to provide security for its intended environment in order to avoid security attacks. A password, regardless of format, is a secret a user needs to use in order to to grant access to a system or device. A password should have certain features in order to be secure:

      \begin{itemize}
        \item The password should be hard to guess, meaning that the password should have a high entropy,
        \item The password should be easy to remember for users, and 
        \item The password should remain a secret for the intended user.
      \end{itemize}

    When we talk about security, we often talk about if a password scheme is ``crackable'', meaning that the a password are guessable. When a password scheme is measured to be ``hard to guess'', it is normal to measure the strength of the password scheme in terms of its entropy. The password strength is measured in terms of information entropy, measured in bits. Instead of measuring the security of a password in number of guesses needed to guess the password, we use the base-2 logarithm of the number of guesses, which is the number of ``entropy bits'' in a password. We use the notation L for the length of the symbols in the password, and they are chosen from a set of N possible symbols. The formula for password entropy are:

      \begin{equation}
        Password Entropy = log_{2}(N^{L})
      \end{equation}

    When we say that a password scheme is easy to use it is normal to measure the success rate when writing and remembering a password, e.g. how long it takes for users to write and remember their passwords. When a password is easy to remember, it often refers to a password schemes ability to maintain its usability. As stated, a password that have a higher entropy are harder to guess, but are often obtained by making long passwords. It is well known that humans have a hard time remembering long and complex passwords. Therefore it is important that a password scheme are supporting the users to make passwords that they can remember, and also are secure. 

    When users make their password it should only remain a secret that the intended user know about. A password scheme that lack support of usability often make people do actions like make simple passwords that are easy to remember, but also easy to guess, or even write down and use the same passwords across multiple systems. 

    All of the key security aspects are important to understand when you are studying password mechanisms. The key aspects will be used throughout this thesis in order to be able to evaluate and read research focusing passwords. 



 %  \section{Passphrase and PIN's vs. graphical passwords}
 %  \section{A password are more then just a password}

 %    If you take a walk in the street and ask a random person ``what is a password?'', 
 %    you probably get the aswer ``its letters and digits''. Passwords are so much more than just letter and digits. 

 %    Nowadays everything we do require you to keep this secret called a password. Your work, you're social life, 
 %    even you're private life is forcing you to keep track of passwords. How do you keep track of all of them?
 %    You probably keep the same password at many places. 
 %    \subsection{Theoretical Password Space}
 %    \subsection{Practical Password Space}

 %  \section{Relevant Data Collection Methods}
 %    In this section I will explore different methods for collecting data. It will give a brief summary of the 
 %    method as well as summary and discussion of the different methods at the end. 
 %    \subsection{Android Unlock Patterns Games}
 %    \subsection{Relevant User Studies}
 %    \subsection{Summary of Methods}
 %  \section{Information gathering}
	% \section{Psycology and passwords}
	% \section{Graphical passwords}
	% \section{Android Unlock Pattern}
