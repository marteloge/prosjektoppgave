  \chapter{An Introduction to Authentication Mechanisms}
  \label{chap:background}
  
    This chapter is an introduction to authentication mechanisms. Section~\ref{sec:authentication} is an introduction to the categorization of existing authentication mechanisms. The different authentication mechanisms will provide a description, as well as it attached benefits and drawbacks of using a particular authentication mechanism. Section~\ref{sec:entropy} provides an introduction to fundamental aspects of security and authentication. Section~\ref{sec:shortcomings} is an evaluation of the text-based authentication and its shortcomings.

  \clearpage

  \section{Authentication} \label{sec:authentication}

  Authentication is the process of verifying whether a particular individual or a device should be granted access to a system or application running on a device \cite{IPAS}, e.g. verifying that you are the person that you claim to be.

  There are various authentication schemes described in the literature, but they can all be grouped by the following characteristics:

    \begin{itemize}
      \item Who you are
      \item Something you have, and
      \item Something you know
    \end{itemize}

    \subsection{Biometric Authentication}
    Biometric authentication has the characteristics of ``who you are''. Biometric authentication refers to verify a person's identity based on physical or behavioral characteristics of an individual \cite{biometrics, biometrics2}. Biometric authentication is different from other authentications schemes because:

      \begin{itemize}
        \item the biometric password cannot be lost nor forgotten,
        \item biometric passwords tends to be difficult to copy, share and distribute, and 
        \item the person being authenticated needs to be present in the authentication process
      \end{itemize} 

    Physiological biometrics uses the physiological characteristics of an individual in the authentication process. The verification uses unique characteristics of a human, e.g. physical parts of the body that are unique as fingerprints, face, iris, hand and finger geometry, and DNA. Behavioral biometrics analyzes how a person performs different activities, e.g. applies pattern recognition techniques for activities like keyboard writing, talking and handwriting.

    The benefits achieved by using biometric authentication is that each password is unique and only connected to one person. When using an authentication scheme that requires the user to use something they know in the authentication process, it might be forgotten or shared. Biometric authentication cannot be shared, nor copied. The user does also not need to remember the biometric password because it is a part of you.

    There is also some drawbacks with biometric authentication. The implementation of biometric authentication is more complicated to implement because of the hardware needed. Other aspects of biometric authentication are the reliability. Many of the existing equipment used at, for instance, an airport only use images of the finger, and many do not detect if a real fingerprint is used. The same is used for face recognition where it is only used algorithms for authenticate the image of the face. In the case of fraud, you can categorize it as identity fraud because a part of you used for authentication is stolen. It is, therefore, a high responsibility for owners of a system using biometric authentication to store the data in a safe way.

    \subsection{Token-based Authentication}
    In a token-based authentication process the user uses ``something you have'' that is often stored on a physical device. Token-based authentication is often combined with a ``something you know'', making a strong authentication by combining two or more authentication characteristics. In many banking systems you have to use more than just something you know, but also ``something you have'', like a one-time password to pass the authentication process. The one-time password is a password that is generated on the physical device. This is an extra layer of security, because even if someone steals or know your password, he or she still can't get access to your banking account because he or she also would need your security token.

    One drawback of using token-based authentication is that the user must carry the token with them. Without the token the users are not able to be authenticated. 

    \subsection{Knowledge-based Authentication}
    ``Something you know'' is often used in the classical login situation where the user has to remember a username/password to get access to the system or device. Some of the commonly used passwords schemes are PIN's, alphanumeric passwords and graphical passwords that all are passwords with the characteristic of ``something you know''.

    One of benefits by using knowledge-based authentication is that this is the authentication type that is mostly used for authentication. Since it is used by most systems, people are also familiar with how it works. The technical solutions do also not require any hardware, and can be implemented at a low cost. 

    One drawback is that the users needs to remember the password used in the authentication. Since the knowledge-based authentication is used in many systems, especially on Internet, the users are also required to remember all the various passwords on different systems. Requiring the users to remember the password often causes the users to select simple password that is easy to remember, and therefore might be easy to guess. 

    There are three different knowledge-based authentication types:

      {\bf PIN's} Personal identification number is a numeric password. The PIN was first introduced in the first ATM in London in 1967 as an efficient way for the banks to authenticate their customers \cite{Bonneau1}. PINS are often a four-digit number. The benefit with a PIN is that it is a short sequence. The drawback is that people often tend to chose a sequence of number that are easily memorized. A known selection of PIN is to chose a date that is easily remembered like the date of birth. Such choices are well known of attackers, and such PIN codes can easily be guessed. 

      {\bf Alphanumeric passwords}
      The word ``Alphanumeric'' is a composition of the phrase ``alpha'' (as in alphabet), and ``numeric''(as in numbers). The alphanumeric password may also contain special characters, so in short an alphanumeric password is a mix of all writable characters. This is rapid used in systems using a combination for username and password. One drawback is that, like with PINs, people often choose a sequence of characters that is connected to the person. A know strategy for passwords selection is to choose words that is associated with the system, or words that are closely related to the person. 

      Systems using alphanumeric passwords often requires the users to change the password within specified intervals. This often cause users to choose a simple password and only make a small change to the password when a change is required. If an attacker know the password policy, it is possible to make a dictionary with likely used words and characters. 

      {\bf Graphical passwords}
      A graphical password has the characteristics of ``something you know'', but instead of using letters and numbers it uses graphical elements. Graphical passwords were proposed as an alternative to PINs and alphanumeric values because humans tends to remember graphical elements better than letters and numbers. A variety of graphical passwords schemes has been created over the past years. Biddle et al. have collected research of the past decade on graphical password schemes \cite{Biddle}, dividing the schemes into three categories:

      \begin{itemize}
        \item Recall-based authentication
        \item Recognition-based authentication, and 
        \item Cued-recall authentication.
      \end{itemize}
      \todo{Bruke disse faguttrykkene mer i state of the art study}

    Recall-based graphical passwords are often referred to as drawmetric systems \cite{DeAngeli} because the user are are reproducing a secret drawing. The password is usually drawn in a grid or a blank canvas, requiring the user to reproduce the secret password from its memory.

    Recognition-based passwords are often referred to as cognometric systems \cite{DeAngeli} because the user recall a secret drawing or sequence of drawings.

    Cued-recall is often referred to locimetric systems \cite{DeAngeli}. With cued-recall authentication typically require the users to remember and target a particular location within and image. This is a version of a recall-based authentication but helps the user with the recall by showing an image and not just a grid or canvas. It is also different from the recognition-based approach because the user needs to identify specific locations in the picture as a whole. 
    
  \section{Key Security Aspects in Authentication} \label{sec:entropy}
  In order to be able to evaluate the security of different password schemes, this section will give a brief introduction to key security aspects of knowledge-based authentication, and hereafter called passwords. In terms of security, the primary goal of authentication is to provide protection for its intended environment in order to avoid security attacks. A password, regardless of format, is a secret a user needs to use in order to to grant access to the system or device. A password should have certain features in order to be secure:

    \begin{itemize}
      \item The password should be hard to guess, meaning that the password should have a high entropy,
      \item The password should be easy to remember for users, and 
      \item The password should remain a secret for the intended user.
    \end{itemize}

  When we talk about security, we often talk about if a password scheme is ``crackable'', meaning that the a password are guessable. When a password scheme is measured to be ``hard to guess'', it is normal to measure the strength of the password scheme in terms of its entropy. The password strength is measured in terms of information entropy, measured in bits. Instead of measuring the security of a password in number of guesses needed to guess the password, we use the base-2 logarithm of the number of guesses, which is the number of ``entropy bits'' in a password. We use the notation L for the length of the symbols in the password, and they are chosen from a set of N possible symbols. The formula for password entropy are:

    \begin{equation}
      Password Entropy = log_{2}(N^{L})
    \end{equation}

  When we say that a password scheme is easy to use it is normal to measure the success rate when writing and remembering a password, e.g. how long it takes for users to write and remember their passwords. When a password is easy to remember, it often refers to a password scheme ability to maintain its usability. As stated, a password that have a higher entropy are harder to guess, but are often obtained by making long passwords. It is well known that humans have a hard time remembering long and complex passwords. Therefore, it is important that a password scheme are supporting the users to make passwords that they can remember, and also are secure.

  When users make their password it should only remain a secret that the intended user know about. A password scheme that lack support of usability often make people do like create simple passwords that are easy to remember, but also easy to guess, or even write down and use the same passwords on multiple systems.

  All of the key security aspects are important to understand when you are studying password mechanisms. The fundamental aspects will be used throughout this thesis in order to be able to evaluate and read research focusing passwords. 

  \section{Shortcomings With Text-Based Authentication} \label{sec:shortcomings}

  User authentication is a central component of security systems. In order to get access to systems, you need to pass the authentication process. Despite the extensive number of options for authentication, text-based passwords remain the most common authentication scheme. The reason they are widely adopted is because they are easy and inexpensive to implement, and users are familiar with the scheme. It is also avoiding the privacy issues raised by biometric authentication and prevent the need for bringing a physical security device that are used in token-based authentication schemes. However, text-based authentication suffers from both security and usability disadvantages. Today, users needs to remember an increasingly number of password, making users adopt bad password habits.

  The term ``habit'' is often a bad thing when talking about security. A habit is often hard to change and are often a predictable pattern. Password reuse is one of the known password habits among users because the human limitation to remembering text-based password. Some users also make simple or meaningful password that are easier to remember, making their passwords vulnerable to attacks. It is a well-known problem that users tend to have an increasingly number of accounts that requires the users to remember yet another number of password on multiple systems and devices. The problem is not just to remember all the password needed, but also remembering which passwords that belong to which account or device. Because of the human capacity for remembering password are causing users to choose weak passwords, as well as reuse the passwords across multiple web pages. In order to understand the shortcomings with text-based passwords, this section will include relevant research on users choices on text-based passwords.

  Password schemes have what is called an empirical password space, and that is the number of possible passwords that a user can make. The problem with many password schemes is that it seems that users don't tend to use the full password space, but only a subset of the possible passwords, e.g. the memorable password space, making the memorable password space less than the empirical password space (Figure~\ref{fig:memorable}). This shows that the security of a password scheme is linked closely to is memorable password space rather than its full password space.

  \begin{figure}[H]
      \centering
      \includegraphics[scale=0.25]{pics/EmpiricalVsPractical.png}
      \caption{Empirical vs. Memorable Password Space}
      \label{fig:memorable}
    \end{figure}

  In a case study of 14.000 Unix passwords, a research group found a 25\% of the passwords were in a group of words forming a dictionary of $3\times10^{6}$ words \cite{UnixPasswords}. This dictionary shows that an attacker can have a relatively high success rate for an attack, despite the fact that there a roughly $2\times10^{14}$ 8-character passwords consisting of digits, and upper case and lower case letters. Due to the limitations of human memory, users often choose passwords that are easier to remember, causing a significant number of user-chosen password to fall into a small dictionary, e.g. practical password space \cite{Tao}. A well-designed dictionary is a tiny subset of the full password space, e.g. theoretical password space, which further can be prioritized according to the likelihood for a password to be chosen. It is, therefore, a commonly stated that the security of a password scheme is related closely to the size of its memorable/practical password space, rather than its theoretical password space. The high success rate of dictionary attack against textual passwords is believed to be strongly related to the recall capabilities of humans and how they choose their passwords, e.g. making meaningful and thus more easily remembered words are selected as passwords.

  One of the first large-scale studies on web password habits was conducted in 2007 by Microsoft research \cite{habits1}. They analyzed web password habits among 544960 internet users over a period of 3 months. The data was collected from a Windows Live Toolbar, and they observed activities like login frequency. They also gathered information about the users age, the strength of the users passwords, as well as number of unique passwords and its use across different URLs. They observed that a typical user have an average of 7 distinct passwords and that an average of 5 of these passwords was re-used on different web pages. An estimate of the average number of account per user was estimated to be 25 accounts per user, but this would probably be higher since it seven years ago. 

  Because of the shortcomings with a text-based authentication, graphical authentication are getting increased attention because it is an alternative to a text-based authentication trying to cope with the memorability and security issues of text-based password. Graphical passwords are using images and visual objects in the authentication instead of text and numeric values. When comparing text against visual objects, the human brain are more capable of remembering images than text. When the human brain are more suited for remembering images, the users might be able to remember more complex passwords that is not easily guessed. To context where an authentication scheme is used have to be evaluated. Graphical passwords might not fit in all systems, but looks like an good alternative on mobile devices where the user interact with an touch sensitive screen. Typing a long text-based password on a small keyboard on a mobile device is not easy. When using graphical passwords are easier to interact with on a small mobile screen.
