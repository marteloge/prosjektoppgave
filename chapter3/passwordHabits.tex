\section{Password Habits Among Users}

  The term ``habit'' is often a bad thing when talking about security. A habit is often hard to change and are often a predictable pattern. Password reuse is one of the known password habits among users. It is a well known problem that users tends to have an increasingly number of account that requires the users to remember yet another number of password across multiple systems and devices. The problem is not just to remember all the password needed, but also remembering which passwords that belongs to which account or device. Because of the human capacity of remembering password are causing users to choose weak passwords, as well as reuse the passwords across multiple web pages. In order to understand the users passwords habits, this section will include relevant research on users passwords habits.

  \subsection{Password Habits Among Web Users}

  One of the first large-scale studies on web password habits was conducted in 2007 by Microsoft research \cite{habits1}. They analyzed web password habits among 544960 web users over a period of 3 months. The data was collected from a Windows Live Toolbar and they observed activities like login frequency. They also collected information about the users age, the strength of the users passwords, as well as number of unique passwords and its use across different URLs. They observed that a normal user have an average of 7 distinct password and that an average of 5 of these password was re-used on different web pages. The estimate on average number of account pr user was estimated to be 25 account pr user, but this would probably be higher since it 7 years ago. 

  Password habits may be different across different subpopulation in cause of different background or culture. In 2012 Joseph Bonneau released a analysis of 70 million passwords from Yahoo! \cite{Bonneau2}. The data is analyzed in terms of guessing rate by using a dictionary attack. The collected data contained 328 subpopulations. The results showed that there was no ``good'' populations among the collected data, but there was a variation in the population. Demographically, the gender had a small effect in the guessing rate, but it showed that age tended to give effect where password strength increases across different age groups. The analysis also showed that language had a significantly effect on the password strength where Indonesian-speaking users were among the weakest subpopulations, and in contrast the German and Korean-speaking users provided relatively stronger passwords. 

  Passwords are not just used in our private life, but also a requirement in a critical concern from a business point of view where the use of authentication for corporate systems, mobile, and room codes plays a major role in normal day of work. In corporate systems users are often promt with the a notification forcing them to change the password in a specified time interval. The problem with this is that user already have problems remembering their passwords as is. A research group conducted a questionnaire survey in a large organization \cite{habits2}. The goal was to get a understanding of password habits in a business point of view. The results showed that the users were prompt with password change 7 times a year causing 68\% of the employees to re-use the same password with a minor change in order to still be able to remembering their passwords.

  \subsection{Security Habits Among Smartphone Users}

  %Intro
  Users are not only dependent on remembering passwords across multiple web pages and systems, but do also need to remember passwords for our small mobile devices. In todays society we're addicted to our mobile devices in our every day life. Mobile devices are not just a communication tool for calling and texting, but also an important tool for every day tasks like doing our work, reading mail, pay our bills and keeping up with our social life. This trend makes our mobile devices vulnerable in terms of security. To avoid unwanted access, smartphones offers different locking mechanisms. The history of locking mechanisms was often a solution solely to prevent accidental use, while current mobile phones require protection in order to secure the potentially vast amount of private data that we keep on our smartphones. Our mobile devices are in rapidly use, leading users to create and reuse shorter passwords and PINs, or no authentication at all. 

  % The time used on unlocking the phone
  In terms of security it is interesting to look at the use of mobile devices and look at the locking habits among users on mobile devices. It is known that services that are rapidly used have weaker password because of the overhead the user needs to spend on typeing their password. In 2014 a group of researchers published a field study of smartphone (un)locking behavior \cite{habits3}. Some of the problems with smartphone users tends to be their rapidly use of their phone. When the device are rabidly use, it results in a lot of time unlocking their phone between every use. In the study they found that there was a significant overhead in the time used of unlocking their phone, where the users participated in the field study used 2.9\% (9\% in the worst case) of their time unlocking their smartphone. 
  
  % The use of locking mechanisms
  Smartphones in use today do not require their users to have a locking mechanism on their smartphone. It is well known that users tends to choose to easiest way out and may result in the choice of not having any locking mechanism at all. Based on the result of the overhead in time used on unlocking their phone, a result may be to take the easiest way out by ignoring the vulnerability of not using a locking mechanism at all. It have been discovered that over 40\% of the users only used a basic ``slide-to-unlock'' mechanism on their smartphone, as well as over 16\% didn't use any locking mechanisms at all \cite{habits3}. This highlights a major bad habit among mobile users. What happens if your mobile is stolen? 

  % Risk vs securtiy
  It is important to understand why people use or not use locking mechanisms on their smartphone. Research have covered that the 46.8\% of the participants agreed or fully agreed that unlocking their phone can be annoying, but at the same time 95.5\% of the somewhat or fully agreed that they liked the idea that their phone was protected \cite{habits3}. This highlights that the users wants to be secure, but there might be a trade-off between the time used to unlock the smartphone vs the security risk.


  \subsection{Graphical Password Habits Among Users}

  One of the popular password schemes on mobile phones are the Android Unlock pattern. It is a graphical password that have been shown to have biases when the password are user-chosen. A research group did a large-scale user study on the Android Unlock Patterns in order to quantify its security \cite{Uellenbeck}. They analyses the biases introduced in the pattern making process and added changes to the scheme in order to avoid the known biases in the password scheme (a description of the Android Unlock Pattern can be found in the background theory). The researchers found that there is a high bias in the pattern selection process, e.g. the upper left corner and three-point long straight lines are very typical selection strategies. If the patterns was uniformly chosen, the probability of starting in the top-left corner should be 11\%, but are instead close to 44\%. Another interesting result is the practical password space used, where about 10\% of all users use less than 190 patterns, while less than 300 patterns capture around 50\% of the whole test population. This shows that a empirical password space are not a representative number when quantifying the security of a password, but we should look at the practical password space, e.g. password that actually are used in practice. The Android Unlock Patterns are a password scheme used on mobile devices. It is important to understand the difference of a password scheme used on a desktop vs. mobile device. As stated earlier a mobile phone have a small screen making it harder to type the password without the standard desktop keyboard. The way that the mobile phone is held, the size of the screen may also impact the way that people write their passwords, but this may need further research to answer. We need to look further into the design space of mobile devices in order to understand how users interact with passwords on mobile devices. 

  %Gender and race
  A study on the PassFace schemes was tested on a group of students \cite{graphical1}. The schemes allows the user to pick a series of different faces. The results showed that there was a high bias in the password selection according to a users demography. When they analyzed how each gender choose their password, the most of the male and female participants chose female faces, and and 60-70\% of the user chose a model over a typical female/male. They also looked into the race of the faces, where the results showed that almost all of the participant chose their own race.