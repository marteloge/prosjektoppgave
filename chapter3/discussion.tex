\section{Discussion and Results}\label{sec:resultsLiteratureStudy}
    
  In the literature review, I have reviewed published research that this project find relevant when looking at graphical passwords. The literature review started with review of graphical password from an historical point of view. This provides an overview of the purposed schemes with the aim of understanding the reason why they were proposed. Each scheme is attempting to either improve drawbacks with earlier published schemes, while some schemes just try to create something new. 

  When going through the history of the published schemes, it was discovered a trade-off between usability and security. Some graphical password schemes solely looks at a new way of designing a password scheme by looking at the usability, the security, and sometimes both. 

  When looking at at text-based passwords and PINs, it is often observed that the passwords are more then just numerical values and letters. I believe that we can look at graphical passwords in the same way; graphical passwords are more then just images and graphical elements. This is the reason why I added the section about psychology and human factors in Section~\ref{sec:humanfactors} in order to understand users choice in graphical passwords. When looking back at Section~\ref{sec:usability}, user selects passwords that they are able to recall by associate the selected password with something they know or are. 

  What is not yet answered in the domain of graphical passwords? Of the published research that included the keyword ``human factors'', only a few of them actually looked at users choice in graphical passwords based on who they are. Most of them only looked at user's choice of passwords in general.

  Smartphones are in rapid use, and is a part of many users everyday life performing their daily tasks. The smartphone is also an interesting platform for graphical password because it of the touch sensitive screen. When conducting research I wanted to look further into a graphical password scheme that actually is in use. The Android Unlock pattern is a well known graphical password scheme used on smartphone. As this research is familiar with, there is only published one paper looking at the schemes security \cite{Uellenbeck}. As stated, I believe that graphical passwords are more than just images and graphical elements. The research that inspired me only looked at what patterns users in general selected. Are we able to add a new dimension to an analysis of users choice in graphical password based on who they are? 

  The issues that needs to be faced when looking at ALP (data collection, no data source available).

  When collecting relevant published research, it was used different combinations of keywords at a selection of publication sites. The procedure for the literature is described in Section~\ref{sec:methodliteraturereview}.
  The majority of the references are published in accepted forums for research. 

  I started reading a summery paper that summarized the twelve first years of graphical passwords \cite{Biddle}. This gave me an historical overview of graphical passwords, as well as providing a long list of relevant research in its bibliographical list. I used this list to find the source of the different graphical passwords schemes, as well as making a list of relevant keywords to use in searching after other relevant research. The list of keywords is found in Table~\ref{tab:keywords}.

    \begin{table}
      \centering
      \begin{tabular}{| p{4cm} | p{4cm} | p{4cm} | }
        \hline
        Security & Usability & Authentication   \\ \hline
        Graphical Passwords & Human Factors & Mobile Security  \\ \hline
        Dictionary Attack & Psychology & Human Behavior \\ \hline
        Memorability & Pattern & Design \\ \hline
      \end{tabular}
      \caption{Keywords used in search}
      \label{tab:keywords}
    \end{table}

  By combing these keywords with logical operators like ``AND'' and ``OR'', I retried a long list of relevant research. For keeping track of the foundings, I used a program called ``PaperPile''. Whenever I found a relevant paper, I added this to PaperPile and tagged it with the keywords stated in the papers. In total, i retrieved a list of 123 relevant papers, whereof 72 was qualified according to different quality attributes that. Out of the 72 papers, 29 papers are included in this literature review.

  In order to get an overview of the references used in the literature, Table~\ref{tab:frequencyKeywords} summarized the number of keywords in different papers used. 

  \begin{table}
      \centering
      \begin{tabular}{| p{1.5cm} | p{1.5cm} | p{1.5cm} | p{1.5cm} | p{1.5 cm} | p{1.5cm} |}
        \hline
        {\bf Security} & 11 & & & & \\ \hline
        {\bf Usability} & 4 & 8 & & & \\ \hline
        {\bf Human factors} & 5 & 2 & 7 & & \\ \hline
        {\bf Graphical Passwords} & 10 & 8 & 7 & 22 & \\ \hline
        {\bf Mobile} & 3 & 1 & 2 & 5 & 5 \\ \hline
         & {\bf Security} & {\bf Usability} & {\bf Human factors} & {\bf Graphical Passwords} & {\bf Mobile} \\ \hline
      \end{tabular}
      \caption{Keywords used in search}
      \label{tab:frequencyKeywords}
    \end{table}



  \todo{Sjekke tallet på antall papers som er blitt inkludert i literature review}








    \todo{Lage en tabell som oppsummerer forskningen}


    % Utfordringen med passord
  % Hvorfor er ikke passordtypene vi har gode nok?

  % Hva betyr det at et passord er knekkbart?

  % hvorfor er dette med passord et viktig tema som man enda ikke anser som løst?
  % ---> Menneskelige faktorer i sikkerhetssammenheng

  % - Det jeg savner at du skriver noe om er utfordringen med passord. Hva søker man etter når man lager nye passordmekansimer? (entropi, vanskelig å gjette, lett å huske etc). Hvorfor er ikke de typene vi har gode nok? Hva betyr det at et passord er knekkbart? Kort oppsummert: hvorfor er dette med passord et viktig tema som man enda ikke anser som løst?
