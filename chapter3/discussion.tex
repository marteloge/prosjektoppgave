\section{Discussion and Results}\label{sec:resultsLiteratureStudy}

  When collecting relevant published research, it was used different combinations of keywords in a selection of publication sites. The procedure for the literature review is described in Section~\ref{sec:methodliteraturereview}.

  I started reading a summary paper that summarized the twelve first years of graphical passwords \cite{Biddle}. This gave me a historical overview of graphical passwords, as well as providing an extensive list of relevant research in its bibliographical list. I used this list to find the source of the different graphical passwords schemes, as well as making a list of relevant keywords use in searching for other relevant research. A list of keywords is found in Table~\ref{tab:keywords}.

    \begin{table}[H]
      \centering
      \begin{tabular}{| p{4cm} | p{3cm} | p{4cm} | }
        \hline
        Security & Usability & Authentication   \\ \hline
        Graphical Passwords & Human Factors & Mobile Security  \\ \hline
        Dictionary Attack & Psychology & Human Behavior \\ \hline
        Memorability & Pattern & Design \\ \hline
      \end{tabular}
      \caption{Keywords used in search}
      \label{tab:keywords}
    \end{table}

  By combing these keywords with logical operators like ``AND'' and ``OR'', I retried a long list of relevant research. For keeping track of the relevant research found, I used a program called ``PaperPile'' to store the research. Whenever I found a related paper, I added this to PaperPile and tagged it with the keywords stated in the documents. In total, I retrieved a list of 123 relevant papers, whereof 85 was qualified according to different quality attributes that are described in Table~\ref{tab:QualityCriteria}. Out of the 85 papers, 29 papers are included in the literature review.

  In order to get an overview of the references used in the literature review, Table~\ref{tab:frequencyKeywords} summarizes keywords utilized in the cited literature. The table does not include all keywords, but a selection of the most frequently used keywords that are relevant to the review. The numbers in the table can be interpreted as: there is 11 out of the included research that includes the keyword ``Security'', and 4 of them contains on both ``Security'' and ``Usability''. 

  \begin{table}[H]
      \centering
      \begin{tabular}{| p{4.7cm} | p{1cm} | p{1cm} | p{1cm} | p{1cm} | p{1cm} |}
        \hline
        {\bf Security (S)} & 11 & & & & \\ \hline
        {\bf Usability (U)} & 4 & 8 & & & \\ \hline
        {\bf Human factors (HF)} & 5 & 2 & 7 & & \\ \hline
        {\bf Graphical Passwords (GP)} & 10 & 8 & 7 & 22 & \\ \hline
        {\bf Mobile (M)} & 3 & 1 & 2 & 5 & 5 \\ \hline
         & {\bf S} & {\bf U} & {\bf HF} & {\bf GP} & {\bf M} \\ \hline
      \end{tabular}
      \caption{Keywords in included research}
      \label{tab:frequencyKeywords}
    \end{table}
    
  In the literature review, I have reviewed published research that this project find relevant when looking at graphical passwords. The literature review started with a review of graphical password from a historical point of view. This provides an overview of the purposed schemes with the aim of understanding the reason they were proposed. Each scheme is attempting to either improve drawbacks with earlier published schemes while some schemes just try to create something new.

  When going through the history of the published schemes, it was discovered a trade-off between usability and security. Some graphical password schemes solely look at a new way of designing a password scheme by looking at the usability, the security, and sometimes both.

  When looking for text-based passwords and PINs, it is often observed that the passwords are more than just numerical values and letters. I believe that we can look at graphical passwords in the same way; graphical passwords are more than just images and graphical elements. This is the reason I added a section about psychology and human factors in Section~\ref{sec:humanfactors} in order to understand user's choice in graphical passwords. When looking back at Section~\ref{sec:usability}, user selects passwords that they can recall by associate the selected password with something they know or are.

  What is not yet answered in the domain of graphical passwords? Of the published research that included the keyword ``human factors'', only a few of them actually looked at users choice in graphical passwords based on who the users are. Most of them only looked at user's selection of passwords in general.

  Smartphones are in rapid use and is a part of many users everyday life used to perform their daily tasks. The smartphone is also an interesting platform for graphical password because it of the touch sensitive screen. When conducting research, I wanted to look further into a graphical password scheme that is in use. The Android Unlock pattern is a well-known graphical password scheme used on smartphones. As this research is familiar with, there is only published one paper looking at the schemes security \cite{Uellenbeck}. As stated, I believe that graphical passwords are more than just images and graphical elements. The research that inspired me only looked at what patterns users in general selected. Are we able to add a new dimension to an analysis of users choice in graphical password based on who they are? 


