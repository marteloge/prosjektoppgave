
\chapter{Literature Study}

  \section{Diverse}

    

  \section{Mobile Devices}

  When talking about graphical password, they are not widely adopted on mobile devices. There are still some graphical passwords that are made for mobile devices, and this section will give a state-of-the art on graphical passwords for mobile devices.

  Our mobile devices are used for different authentication task. In the background theory, it was stated that we had mainly three different authentication schemes, e.g. ``something you know'', ``something you have'', and ``something you are''.

  As stated earlier, the mobile phone have emerged as a good platform for graphical passwords because it is easier to input on touchscreen as a contrast to text-based passwords. Graphical passwords on mobile devices seems as a natural fit, as they often require direct manipulation of visual elements. 

  \section{Graphical Password Schemes}

    In 1999, Jermyn et al. \cite{Jermyn} suggested a new graphical password scheme called DAS (Draw-a-secret). Draw-A-Secret (DAS) was the first recall-based graphical password scheme proposed. The motivation for the graphical password scheme was that graphical input devices enables the user to decouple the position of inputs from the temporal order in which they occur, and shows that the decoupling can be used to genereate passwords that have a larger and more memorable password space. In order to make a more memorable password, the research group argumented that the DAS was more secure than text-based passwords. 

