
\chapter{State of the Art - Graphical Passwords}

  % Utfordringen med passord
  % Hvorfor er ikke passordtypene vi har gode nok?

  % Hva betyr det at et passord er knekkbart?

  % hvorfor er dette med passord et viktig tema som man enda ikke anser som løst?
  % ---> Menneskelige faktorer i sikkerhetssammenheng

  % - Det jeg savner at du skriver noe om er utfordringen med passord. Hva søker man etter når man lager nye passordmekansimer? (entropi, vanskelig å gjette, lett å huske etc). Hvorfor er ikke de typene vi har gode nok? Hva betyr det at et passord er knekkbart? Kort oppsummert: hvorfor er dette med passord et viktig tema som man enda ikke anser som løst?

  \section{Diverse}

    {\bf Bias:} When a user are able to make their own password, the password that is created is often influenced by different biases.  When we are talking about a bias in terms of passwords, a the password making process can be influenced by biases like the demography of a person or the visualization of the password scheme. 

    {\bf Deja Vu vs. PIN and passwords:} Our user study shows that 90\% of all participants succeeded in the authentication tests using Deja Vu while only about 70\% succeeded using pass words and PINS \cite{DejaVu}.
    
    {\bf Shortcomings with text-based password schemes:} \cite{DejaVu}

    {\bf Mobile Devices: } When talking about graphical password, they are not widely adopted on mobile devices. There are still some graphical passwords that are made for mobile devices, and this section will give a state-of-the art on graphical passwords for mobile devices.
    Our mobile devices are used for different authentication task. In the background theory, it was stated that we had mainly three different authentication schemes, e.g. ``something you know'', ``something you have'', and ``something you are''.
    As stated earlier, the mobile phone have emerged as a good platform for graphical passwords because it is easier to input on touchscreen as a contrast to text-based passwords. Graphical passwords on mobile devices seems as a natural fit, as they often require direct manipulation of visual elements. 

