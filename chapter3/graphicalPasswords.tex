\section{Graphical Passwords}

% Må få frem hvilke grafiske passord som er laget, samt se på hvilke problemer de løser/ikke løser
% Kommentar fra Lillian: Hva søker man etter når man lager nye passordmekanismer? (entropi, vanskelig å gjette, lett å huske)

  Like text-based passwords schemes, graphical password schemes are also a knowledge-based authentication scheme, e.g. ``something you know'' that are described in the background theory. Since it all started around 1996, there have been many suggestions for graphical password schemes. When a new password scheme are proposed, there are several aspects of password that needs to be considered. A password scheme needs to be secure in terms of entropy, it needs to be hard to guess and it also needs to be easy to use. This section will give a brief introduction to the history of research published on graphical passwords. This is important to know because each scheme is trying to improve different aspects of graphical password, giving us a detailed understanding of todays situation. 

  The idea of graphical passwords was originally described by Greg Blonder in 1996 \cite{Blonder}. The graphical password scheme proposed was requiring the users to tap on a selection of points on a predefined image in order to pass the authentication process. This was just a proposal, and did not further explore the power of graphical passwords, nor analyzed the security aspects of the proposal. 

  In 1999, Jermyn et al. \cite{Jermyn} suggested a new graphical password scheme called ``DAS'' (Draw-a-secret). Draw-A-Secret (DAS) was the first recall-based graphical password scheme proposed. The motivation for the graphical password scheme was that graphical input devices enables the user to decouple the position of inputs from the temporal order in which they occur, and shows that the decoupling can be used to generate passwords that have a larger and more memorable password space. In order to make a more memorable password, the research group argued that the DAS was more secure than text-based passwords because the users were able to remember longer and more complex passwords. After the DAS scheme was published, Dunphy and Yan \cite{BDAS} added an extra background image, called ``BDAS'', in order to encourage their users to make more complex passwords.

  In 2000, Dhamija and Perrig \cite{DejaVu},created a new password scheme called ``Deja Vu''. The password scheme was based on the hash visualization technique \cite{HashVisualization}. The users are asked to select a sequence of images from a random set of images that are generated by a program. They wanted to make a graphical password scheme that solved some of the shortcomings with recall-based authentication like PIN's and text-based passwords. Deja vu should purely rely on recognition rather than recall, and it should be hard to write down and share the password with others. The randomly generated pictures based on the hash visualization technique makes it hard to share the password since the pictures is hard to recreate, but are easy to remember. 

  ``Passfaces'' is a graphical password scheme developed by Real User Corporation that was founded in 2000 \cite{passface}. The authentication procedure allows the users to first select four images that are a visualization on human faces, and the user get authenticated by identifying their four faces. The scheme exploits the advantage that people are good at recognizing people, so when users choose the human faces, they can recognize the characteristics of the faces.

  In 2002, Goldberg et al. \cite{PassDoodle} tried to make a graphical password scheme that combined both text an images called ``PassDoodle''. This is a graphical password combined of handwritten text. Their study concluded that users were able to remember complete doodle images as accurately as text-based passwords.

  In 2004, Davis et al. did a comparison of a light version of the ``PassFace'' and a new graphical password scheme called ``Story'' \cite{Davis}. The Story scheme is making the users choosing images making a story. The background for the scheme was to support their users of remembering their passwords by making a memorable story of images. The story that was made were needed to be recalled in a correct order. To aid memorability, users were instructed to mentally construct a story to connect the everyday images in the set. 

  In 2005 Wiedenbeck and Blonder made a graphical password scheme called ``PassPoints'' \cite{PassPoints} that is an  extension of the Blonder's \cite{Blonder} idea by eliminating the boundaries and allowing arbitrary images to be used. They evaluated their password scheme by testing the scheme on human users. The results showed that PassPoint were a promising scheme with respect to memorability because of the low error rate and low clicking rate. 
  The aim of this study was to get an understanding of how different images affected user performance in authentication with a graphical password scheme. The preliminary result showed suggested that images may support memorability in graphical password schemes. 

  In 2006, a research wanted to address the problem with graphical passwords and the shoulder surfing problem. They called their password scheme ``Convex Hull Click'' (CHC) \cite{Wiedenbeck} that allows the user to prove knowledge of the graphical password in secure and insecure location because they made the scheme in a way that users don't directly click on their password images, making it hard for attackers to do perform shoulder surfing. In CHC the windows shows a list of small icons. In the authentication process, the user needs to recognize some minimum number of their chosen password images, or ``pass-icons'', out of a large number of randomly places icons. This step are presented in a sequence, and if the user responds correctly every time, the user pass the authentication.
  
  In 2007, Tao and Adams \cite{Tao}, created a new proposal for a new graphical password scheme called ``Pass-Go''. The Pass-Go scheme is inspired by the old Chinese game, Go, where users selects intersections on a grid to maker their password. This was one of the first largest user studies on graphical passwords, and was made in order to improve the usability of graphical passwords. They try to emphasize that the usability of a graphical password scheme will increase the memorability of graphical password, causing the password scheme to be more secure.  

  Graphical passwords are also implemented on mobile devices, like the ``Android Unlock Pattern'', that is an mini version of the ``Pass-Go'' deployed on Google Android smartphones. ``PatternLock'' is a similar system that are available for Blackberry. Rather than entering a 4 digit PIN or a text-based password, the user enters a touch-drawn password on a $3\times3$ grid.

  Graphical passwords are still not widely adopted, but there are still new graphical password scheme being proposed. Recently published  graphical password schemes are GeoPass \cite{GeoPass} and Picassopass \cite{PicassoPass}. Geopass is uses a digital map for the authentication phase, where the user chooses a specific location as their password. Picassopass is a graphical password scheme that are presenting a password using a dynamic layered combination of graphical elements. The users can make a story that assists the user in the recognition of the graphical elements.

 
