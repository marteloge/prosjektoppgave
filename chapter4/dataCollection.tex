\section{Data Collection} \label{sec:datacollection}
  
  This section will include information about the data collection process in detail. Section~\ref{sec:formodadministration} will include information on the administrative part of the questionnaire. Section~\ref{sec:questions} will be a list of the questions contained in the questionnaire and the layout and structure of the questionnaire will further be described in Section\ref{sec:layout}.

  \subsection{Form of Administration} \label{sec:formodadministration}

  There are two different ways of administrate the questionnaire, ``self-administered'' and ``researcher-administered''. This questionnaire is a``Self-administered'', meaning that volunteered respondents complete the questionnaire without me being present. This choice is made due time saved, as well as getting enough data. A ``researcher-administrated'' questionnaire would simply be too time-consuming, and there is a chance that the required sample size would not be achieved.

  When sending the same questionnaire over the Internet, it eliminates the risk of introducing bias by asking the respondents the questions in a different manner, and bias introduced by tone and body language. When collecting passwords, the questionnaire needs to be untraceable back to the respondents due to ethical considerations. A questionnaire over the Internet would make it easier for respondents to share patterns, but it also introduce the problem with the control of the responses and its reliability. The questionnaire needs to be carefully designed in order to reduce possible biases. To get the data in a standardized format, the questionnaire will include closed question and will generate quantitative data. The asked questions will further be described in the next subsection.

  \subsection{Question and Responses}\label{sec:questions}

  In this section, we will look further into the questions added to the questionnaire as well as the responses to each question. Further discussion of the questions and responses is included in Section~\ref{sec:wireframes}.
  The questionnaire is divided into two parts: pattern selection and respondent information. 

    \subsection*{Pattern selection}
    The pattern selection is divided into three different tasks. The reason for doing this is getting more data, as well as obtaining patterns with various security levels. {\it First}, there is a need for a large sample, and by asking the respondents to make three different patterns gives this research more patterns to analyze. {\it Second}, the patterns now add a new dimension by categorizing the patterns into three security levels: low, medium, and high. {\it Third}, it reduces the risk of people making pattern without adding an effort into the pattern creation. This research has some concerns with the possibility that users might be making patterns that they normally wouldn't use in the real world. By adding a context for the patterns that is being created, it might reduce some of the effortless attempts that some respondents would do.

    When coping with the possibility of users selecting patterns without an effort to make real patterns, there is a possibility to add a sequence at the end of the questionnaire where the respondents have to reenter their selected patterns. The problem with such an approach is that it might be likely that the respondents make three equal patterns that are too easy to guess because they are afraid to not remember them in the end. Therefore, this questionnaire will not include any reentering of the patterns.

    When asking all users with a smartphone, there might be users that never have used the Android Unlock Pattern before. It is, therefore, added the possibility to enter a training mode before entering the patterns. For experienced users, they can skip the training mode and go directly to the pattern creation.

    The respondents are asked for making three patterns:

      {\bf P1:} {\it Make pattern to protect one of your shopping accounts}

      {\bf P2:} {\it Make a pattern to protect your smartphone}

      {\bf P3:} {\it Make a pattern to protect your banking account}

    \clearpage
    \subsection*{Information about the respondents} 

    When the respondents created three various patterns, the respondents are asked different questions about their human properties, as well as some information about their smartphones (Figure~\ref{tab:questions}). 

    The ordering of the questions is arranged according to the importance of the questions. This ordering is added in case if some respondents choose to terminate the questionnaire before completion of the questionnaire. The questions are selected from the review of human properties (Section~\ref{sec:datarequirements}), as well as some questions that includes information about the smartphone used for answering the questionnaire. Further discussion of each question will be included in Section~\ref{sec:wireframes}.

    \begin{longtable}{| p{1cm} | m{6.5cm} | m{3.5cm} |}
      \hline
      {\bf \#} & {\bf Question} & {\bf Alternatives} \\ \hline
      {\bf Q1.1} & 
      {\it How would you categorize the size of your hand?} & 
      \begin{itemize}
        \item Small
        \item Medium
        \item Large
        \item Xtra Large
      \end{itemize} 
      \\ \hline

      {\bf Q1.2} & 
      {\it How would you categorize the screen size of your smartphone?} &
      \begin{itemize}
        \item Small
        \item Medium
        \item Large
      \end{itemize} 
      \\ \hline

      {\bf Q1.3} & 
      {\it Which hand did you hold you smartphone in when you created the patterns?} & 
      \begin{itemize}
        \item Left-hand
        \item Right-hand
      \end{itemize} \\ \hline

      {\bf Q1.4} & 
      {\it What finger did you use when making the patterns?} &
      \begin{itemize}
        \item Thumb
        \item Forefinger
      \end{itemize}  \\ \hline

      {\bf Q1.5} & 
      {\it What is you main reading/writing orientation?} &
      \begin{itemize}
        \item Left-to-right
        \item Right-to-left
        \item Top-to-bottom, left-to-right
      \end{itemize} \\ \hline

      {\bf Q1.6} & 
      {\it What is your gender?} &
      \begin{itemize}
        \item Male
        \item Female
      \end{itemize} \\ \hline

      {\bf Q1.7} & 
      {\it What is your age?} &
      Specific age of the respondent \\ \hline

      {\bf Q1.8} & 
      {\it What is your nationality?} &
      Nationality of the respondent \\ \hline

      {\bf Q1.9} & 
      {\it Have you ever used the Android Unlock Patten?} &
      \begin{itemize}
        \item Yes
        \item No
      \end{itemize} \\ \hline

      {\bf Q1.10} & 
      {\it Do you use screen lock to protect you smartphone?} &
      \begin{itemize}
        \item Yes
        \item No
      \end{itemize} \\ \hline

      {\bf Q1.11} & 
      {\it What kind of screen lock do you currently use on your smartphone?} &
      \begin{itemize}
        \item Pattern
        \item Fingerprint
        \item PIN
        \item Slide-to-unlock
        \item Other
      \end{itemize} \\ \hline
    \caption{Questions included in the questionnaire}
    \label{tab:questions}
    \end{longtable}

  \clearpage
  \subsection{Layout and Structure}\label{sec:layout}

  In this section, we will describe the layout and structure of the questionnaire. The questionnaire is divided into four parts: introduction, pattern training, pattern creation and collection of information about the respondents and device used. 

    \subsubsection*{Introduction}
    Before the questionnaire starts there is given some information about the research, the use of the collected data, contact information, and privacy concerns. The information is an important part of the questionnaire because it can be essential to get people to participate. It needs to provide sufficient information about the background of the research. The contact Information is necessary because if any respondents have any questions, or unsure if they want to participate, they can see that the contact information is connected to NTNU. 

    \subsubsection*{Pattern training}
    After the information about the research, the respondent will be given the possibility to practice before entering the patterns. The training is optional, and experienced respondents can skip the training and is  them directly to the pattern creation.

    \subsubsection*{Pattern creation}
    After the information and training, the respondents will be asked to make a pattern for three different categories of security risks.

    It is a concern that respondents could make patterns that are done quickly without any thoughts. When asking a person to make a ``complex pattern'', people tend to overcompensate and choose rather complicated passwords. This behavior is well known in psychology as ``priming''. By overcompensating and choosing more complex patterns than probably would appear ``in the wild'', would introduce bias in the data because the overcompensated password is not representative. By asking people to make three different patterns, we get an overview of the complexity without directly asking the users to make a password that is hard to guess.

    One concern with asking the users to create the patterns in a predefined order is that it might introduce bias in the data. When a respondent is making the patterns the order might affect how they make the patterns. For example, when a respondent creates a pattern for a shopping account, they may start out with an easy pattern because they might categorize it as a a lower security risk, for example, the banking account. To be sure that the ordering is not introducing bias, we can use a method called a ``Latin Square'' (Figure~\ref{tab:latin}). A Latin square is a table filled with n different symbols in such a way that each symbol occurs exactly once in each row and exactly once in each column.

    We use a Latin Square for ensuring that the ordering do not impact the respondent's choice of patterns. In practice, different respondents will be answering the same questions, but in a different order. The answers with a different ordering will be compared to see if the ordering had any impact on the results. 

    \begin{table}[H]
      \centering
      \begin{tabular}{| m{1cm} | m{1cm} | m{1cm} |}
        \hline
        P1 & P2 & P3 \\ \hline
        P2 & P3 & P1 \\ \hline
        P3 & P1 & P2 \\ \hline
      \end{tabular}
      \caption{Latin Square}
      \label{tab:latin}
    \end{table}

    \subsubsection*{Information about the respondents and the device used}
    After the patterns are entered, the background and demographics of the users will be asked. There is a thought behind the ordering of the different parts of the questionnaire. {\it First}, people might stop before finishing the questionnaire, and is, therefore, desired to collect the patterns first. {\it Second}, the additional data collected might impact the user's choice in patterns. For example, when asking a person about their hand size or the size of the screen, the respondent might be aware that these properties will be impacting their choice of patterns.

