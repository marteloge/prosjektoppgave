\section{Data Collection} \label{sec:datacollection}


  \subsection{Form of Administration} \label{sec:formodadministration}

  This questionnaire is a``Self-administered'', meaning that volunteered respondents completes the questionnaire without me being present. This choice is made due time saved, as well as getting enough data. A ``researcher-administrated'' questionnaire would simply be too time consuming and there is a chance that the required amount of data would not be reached. 
    
  When sending the same questionnaire over the Internet, it eliminates the risk of introducing bias by asking the respondents the questions in a different manner, and bias introduced by tone and body language. When collecting passwords, the questionnaire needs to be untraceable back to the respondents due to ethical considerations. A questionnaire over the Internet would make it easier for respondents to share patterns, but it also introduce the problem with the control of the responses and its reliability. The questionnaire needs to be carefully designed in order to reduce possible biases. To get the data in a standardized format, the questionnaire will include closed-question and will therefore generate quantitative data. The asked questions will further be described in the next subsection

  \subsection{Question and Responses}\label{sec:questions}

    \subsection*{Pattern selection}
    The pattern selection is divided into 3 different tasks. The reason for doing this is getting more data, as well as setting a pattern into a security setting with different security levels. {\it First}, there is a need for a large scale of data, and by asking the respondents to make three different patterns gives this research more data to analyze. {\it Second}, the pattens now adds a new dimension by categorizing the patterns into three security categories: low, medium, and high. 

      {\bf P1:} {\it Make pattern to protect one of your shopping accounts}

      {\bf P2:} {\it Make a pattern to protect your smartphone}

      {\bf P3:} {\it Make a pattern to protect your banking account}

  \clearpage
    \subsection*{Information about the respondents} 

    \begin{longtable}{| p{1cm} | m{6.5cm} | m{3.5cm} |}
      \hline
      {\bf \#} & {\bf Question} & {\bf Alternatives} \\ \hline
      {\bf Q1.1} & 
      {\it How would you categorize the size of your hand?} & 
      \begin{itemize}
        \item Small
        \item Medium
        \item Large
        \item Xtra Large
      \end{itemize} 
      \\ \hline

      {\bf Q1.2} & 
      {\it How would you categorize the screen size of your smartphone?} &
      \begin{itemize}
        \item Small
        \item Medium
        \item Large
      \end{itemize} 
      \\ \hline

      {\bf Q1.3} & 
      {\it Which hand did you hold you smartphone in when you created the patterns?} & 
      \begin{itemize}
        \item Left-hand
        \item Right-hand
      \end{itemize} \\ \hline

      {\bf Q1.4} & 
      {\it What finger did you use when making the patterns?} &
      \begin{itemize}
        \item Thumb
        \item Forefinger
      \end{itemize}  \\ \hline

      {\bf Q1.5} & 
      {\it What is you main reading/writing orientation?} &
      \begin{itemize}
        \item Left-to-right
        \item Right-to-left
        \item Top-to-bottom, left-to-right
      \end{itemize} \\ \hline

      {\bf Q1.6} & 
      {\it What is your gender?} &
      \begin{itemize}
        \item Male
        \item Female
      \end{itemize} \\ \hline

      {\bf Q1.7} & 
      {\it What is your age?} &
      Specific age of the respondent \\ \hline

      {\bf Q1.8} & 
      {\it What is your nationality?} &
      Nationality of the respondent \\ \hline

      {\bf Q1.9} & 
      {\it Have you ever used the Android Unlock Patten?} &
      \begin{itemize}
        \item Yes
        \item No
      \end{itemize} \\ \hline

      {\bf Q1.10} & 
      {\it Do you use screen lock to protect you smartphone?} &
      \begin{itemize}
        \item Yes
        \item No
      \end{itemize} \\ \hline

      {\bf Q1.11} & 
      {\it What kind of screen lock do you currently use on your smartphone?} &
      \begin{itemize}
        \item Pattern
        \item Fingerprint
        \item PIN
        \item Slide-to-unlock
        \item Other
      \end{itemize} \\ \hline
    \caption{Questions included in the questionnaire}
    \end{longtable}

  \clearpage
  \subsection{Layout and Structure}\label{sec:layout}

    Before the questionnaire starts there is given some information about the research, the use of the collected data, and privacy concerns. After the information, the respondent will be given the possibility to practice before entering the patterns. This is optional, and is added for users that is not familiar with the Android Unlock Pattern.

    After the training, the respondents will be asked to make a pattern for three different categories of security risks. It is a concern that respondents could make patterns that is done quickly without any thoughts. When asking a person to make a ``complex pattern'', people tend to overcompensate and choose rather complicated passwords. This behavior is well known in psychology as ``priming''. By overcompensating and choosing rater more complicated patterns than probably would appear ``in the wild'' would introduce bias in the data because the overcompensated password is not representative. By asking people to make three different patterns, we get a overview of the complexity without directly asking the users to make a password that is hard to guess.

    After the patterns is entered, the background and demographics of the users will be asked for. There is a thought behind the ordering of the different parts of the questionnaire. {\it First}, people might stop before finishing the questionnaire, and is therefore desired to collect the patterns first. {\it Second}, the additional data collected might impact the users choice in patterns. For example, when asking a person about their hand size, or the size of the screen, the respondent might be aware that this properties will be impacting their patterns. A such observation form the user could introduce bias in the data.

    This is will be the structure of the questionnaire:

    \begin{enumerate}
      \item {\bf Introduction}
      \item {\bf Pattern training}
      \item {\bf Selection of Pattens}
      \item {\bf Background and Demographics}
    \end{enumerate}

  \subsection{Validity and Reliability}\label{sec:validityandreliability}

    {\bf Content validity}

    {\bf Construct validity}

    {\bf Reliability}

  

    %Forstår de hva som skal svares på? Ville de svart på denne undersøkelsen? Er etiske aspekter brutt? Vil det ta for lang tid?

 

  

  

% Bør si noe om vanskeligheten med å samle inn password og forskjellen på å samle inn lekasje of faktiske passord.
% Hva bruker man mobilen til? Bank, facebook, mail, jobb, etc? Kan man se en sammenheng mellom passord og bruk av mobil. 
% Lage passord for forskjellige risikokategorier.

