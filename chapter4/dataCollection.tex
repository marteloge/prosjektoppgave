\section{Data Collection} \label{sec:datacollection}


  \subsection{Form of Administration} \label{sec:formodadministration}

  This questionnaire is a``Self-administered'', meaning that volunteered respondents completes the questionnaire without me being present. This choice is made due time saved, as well as getting enough data. A ``researcher-administrated'' questionnaire would simply be too time consuming and there is a chance that the required amount of data would not be reached. 
    
  When sending the same questionnaire over the Internet, it eliminates the risk of introducing bias by asking the respondents the questions in a different manner, and bias introduced by tone and body language. When collecting passwords, the questionnaire needs to be untraceable back to the respondents due to ethical considerations. A questionnaire over the Internet would make it easier for respondents to share patterns, but it also introduce the problem with the control of the responses and its reliability. The questionnaire needs to be carefully designed in order to reduce possible biases. To get the data in a standardized format, the questionnaire will include closed-question and will therefore generate quantitative data. The asked questions will further be described in the next subsection

  \subsection{Question Content and Responses}\label{sec:questions}

    \subsection*{Pattern selection}
    The pattern selection is divided into 3 different tasks. The reason for doing this is getting more data, as well as setting a pattern into a security setting with different security levels. {\it First}, there is a need for a large scale of data, and by asking the respondents to make three different patterns gives this research more data to analyze. {\it Second}, the pattens now adds a new dimension by categorizing the patterns into three security categories: low, medium, and high. 

      {\bf P1:} {\it Make pattern to protect one of your shopping accounts}

      {\bf P2:} {\it Make a pattern to protect your smartphone}

      {\bf P3:} {\it Make a pattern to protect your banking account}

    \subsection*{Information about the respondents:} 
    This information be used to group respondents in different subgroups to see if there is different choices in patterns in the subgroups. It it also valuable information to collect to see the diversity of the respondents. Some of the interesting human properties selected is the reading orientation. By studying the way humans read we can might use this property to predict users choice in patterns. A theory is that humans would start their pattern in the same direction as their reading orientation. 

      {\bf Q1.1:} {\it How would you categorize the size of your hand?} \\
      The size of the hand can be used to predict likely chosen patterns. Size of the hand is likely to impact the areas the user is able to reach on the mobile screen. This property can be used together with the screen size of the mobile phone in the analysis. 
        \begin{enumerate*}
          \item[ ] {\bf Alternatives:} {\it Small, Medium, Large, XLarge}
        \end{enumerate*}

      {\bf Q1.2:} {\it How would you categorize the screen size of your smartphone?} \\
      As described earlier, the screen size might impact the users choice of patterns. A big screen will maybe introduce areas of the screen that is hard for the user to reach. It is likely that areas that are hard to reach might not be included in the selected pattern.
        \begin{enumerate*}
          \item[ ] {\bf Alternatives:} {\it Small, Medium, or Large}
        \end{enumerate*}

      {\bf Q1.3:} {\it Which hand did you hold you smartphone in when you created the patterns?} \\
      In the review of human properties (Subsection\ref{sec:datarequirements}), it was stated that is was likely that people that was right-handed to start their pattern i the left-upper corner, and that left-handed respondents would start in the right-upper corner. Instead of asking if the respondents are left- or right-handed, I want to ask what hand the respondents are using when interacting with the smartphone. 
        \begin{enumerate*}
          \item[ ] {\bf Alternatives:} {\it Left-hand or Right-hand}
        \end{enumerate*}

      {\bf Q1.4:} {\it What finger did you use when making the patterns?} \\
        \begin{enumerate*}
          \item[ ] {\bf Alternatives:} {\it Thumb or Forefinger}
        \end{enumerate*}

      {\bf Q1.5:} {\it What is you main reading/writing orientation?} \\
      Instead of making assumptions if a nationality have a specific reading orientation, the question is asked directly. A person from Asia do not necessarily write from top-to-bottom.
        \begin{enumerate*}
          \item[ ] {\bf Alternatives:} {\it Left-to-right horizontally, Right-to-left horizontally, or Top-to-bottom Left-to-right}
        \end{enumerate*}

      {\bf Q1.6:} {\it What is your gender?}
        \begin{enumerate*}
          \item[ ] {\bf Alternatives:} {\it Male or Female}
        \end{enumerate*}

      {\bf Q1.7:} {\it What is your age?} \\
      This question will not be pre-divided into specific groups of age. The alternatives in this question will be a open question where the respondents type their specific age. A grouping of the respondents is done when the data is collected.
        \begin{enumerate*}
          \item[ ] {\bf Alternatives:} {\it The specific age of the respondent}
        \end{enumerate*}

      {\bf Q1.8:} {\it What is your nationality?} \\
        \begin{enumerate*}
          \item[ ] {\bf Alternatives:} {\it The selected nationality of the respondent}
        \end{enumerate*}

      {\bf Q1.9:} {\it Have you ever used the Android Unlock Patten?}
      For example, a person with no experience with the Android Unlock Pattern could may introduce bias because the user would not be familiar with its use and functionality. 
        \begin{enumerate*}
          \item[ ] {\bf Alternatives:} {\it Yes or No}
        \end{enumerate*}

      {\bf Q1.10:} {\it Do you use screen lock to protect you smartphone?}
        \begin{enumerate*}
          \item[ ] {\bf Alternatives:} {\it Yes or No}
        \end{enumerate*}

      {\bf Q1.11:} {\it What kind of screen lock do you currently use on your smartphone?} \\
        \begin{enumerate*}
          \item[ ] {\bf Alternatives:} {\it Unlock Pattern, Fingerprint, PIN, Slide-to-Unlock, Other}
        \end{enumerate*}

      {\bf Q1.12:} {\it Is you mobile operating system ``detected OS''?} \\
      The reason for asking this question is to be able to check if the device used to answer the questionnaire provides the pattern lock security mechanism. 
        \begin{enumerate*}
          \item[ ] {\bf Alternatives:} {\it Android, iOS, Windows, or Other}
        \end{enumerate*}

      {\bf Q1.13:} {\it Do you work with IT and/or security full time or studied IT and/or security?} \\
      To ask the respondents about their profession is not easy. {\it First}, the list of potential professions is a long list. {\it Second}, it might take a lot of time for the user to find a suitable alternative that describes their profession. I want to use this question to see if people with a profession in IT and Security introduce bias in the data. People with a profession in IT or security may try to overcompensate to prove their knowledge, or they will make more complex patterns than people with a other profession. The list of statements will ask if the respondents have worked with or studies IT or security. 
        \begin{enumerate*}
          \item[ ] {\bf Alternatives:} {\it Yes or No}
        \end{enumerate*}

  \clearpage
  \subsection{Layout and Structure}\label{sec:layout}

    Before the questionnaire starts there is given some information about the research, the use of the collected data, and privacy concerns. After the information, the respondent will be given the possibility to practice before entering the patterns. This is optional, and is added for users that is not familiar with the Android Unlock Pattern.

    After the training, the respondents will be asked to make a pattern for three different categories of security risks. It is a concern that respondents could make patterns that is done quickly without any thoughts. When asking a person to make a ``complex pattern'', people tend to overcompensate and choose rather complicated passwords. This behavior is well known in psychology as ``priming''. By overcompensating and choosing rater more complicated patterns than probably would appear ``in the wild'' would introduce bias in the data because the overcompensated password is not representative. By asking people to make three different patterns, we get a overview of the complexity without directly asking the users to make a password that is hard to guess.

    After the patterns is entered, the background and demographics of the users will be asked for. There is a thought behind the ordering of the different parts of the questionnaire. {\it First}, people might stop before finishing the questionnaire, and is therefore desired to collect the patterns first. {\it Second}, the additional data collected might impact the users choice in patterns. For example, when asking a person about their hand size, or the size of the screen, the respondent might be aware that this properties will be impacting their patterns. A such observation form the user could introduce bias in the data.

    This is will be the structure of the questionnaire:

    \begin{enumerate}
      \item {\bf Introduction}
      \item {\bf Pattern training}
      \item {\bf Selection of Pattens}
      \item {\bf Background and Demographics}
    \end{enumerate}

  \subsection{Validity and Reliability}\label{sec:validityandreliability}

    {\bf Content validity}

    {\bf Construct validity}

    {\bf Reliability}

  

    %Forstår de hva som skal svares på? Ville de svart på denne undersøkelsen? Er etiske aspekter brutt? Vil det ta for lang tid?

 

  

  

% Bør si noe om vanskeligheten med å samle inn password og forskjellen på å samle inn lekasje of faktiske passord.
% Hva bruker man mobilen til? Bank, facebook, mail, jobb, etc? Kan man se en sammenheng mellom passord og bruk av mobil. 
% Lage passord for forskjellige risikokategorier.

