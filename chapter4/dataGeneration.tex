\section{Data Collection} \label{sec:datacollection}


  \subsection{Form of Administration} \label{sec:formodadministration}

    This questionnaire is a``Self-administered'', meaning that volunteered respondents completes the questionnaire without me being present. This choice is made due time saved, as well as getting enough data. A ``researcher-administrated'' questionnaire would simple be too time consuming and there is a chance that it would not reach the required amount of collected data needed for the analysis. When sending the same questionnaire over the Internet, it eliminates the risk of introducing bias by asking the respondents the questions in a different manner, and bias introduced by tone and body language. When collecting passwords the questionnaire needs to be untraceable back to the respondents due to ethical considerations. A questionnaire over the Internet would make it easier for respondents to share patterns, but it also introduce the problem with the control of the responses and its reliability. The questionnaire needs to be carefully designed in order to reduce possible biases. To get the data in a standardized format, the questionnaire will include closed-question and will therefore generate quantitative data. 

  \subsection{Question Content and Responses}\label{sec:questions}

    \subsubsection*{Background information}
    To be able to analyze the data, it is a need to know something about the respondents background information that can either be useful in the analysis or be factors that may introduce bias in the data. For example, a person with no experience with the Android could may introduce bias because the user would not be familiar with its use and functionality. There is added question about the type of smart phone that the respondent is in currently in position of. The reason for asking this question is to be able to  check if the device used to answer the questionnaire provides the pattern lock security mechanism. In the analysis of human properties, there was selected factors that may would impact the choice of patterns. One of the properties that is selected is the hand size as well as the mobile size. These two properties would be interesting to analyze together. My own experiences shows that it its hard to reach certain areas of the screen with a small hand and a big mobile screen. \\

    {\bf $Q1.1$:} {\it What kind of smart phone do you currently own?}

    {\bf $Q1.2$:} {\it How would you categorize the screen size on your current smart phone?}

    {\bf $Q1.3$:} {\it Have you ever used the Android Unlock Patten?}

    {\bf $Q1.4$:} {\it What kind of locking mechanism do you currently use on your smart phone?}

    {\bf $Q1.5$:} {\it How would you categorize the size of your hand?}

    {\bf $Q1.6$:} {\it Which hand do you usually use for interacting with your mobile phone?}

    {\bf $Q1.7$:} {\it Do you normally use a locking mechanism on your mobile phone?}

    \subsubsection*{Demographic information}
    Demographic information can be used to group respondents in different subgroups to see if there is different choices in patterns in the subgroups. It it also valuable information to collect to see the diversity of the respondents. Some of the interesting human properties selected is the reading orientation. By studying the way humans read we can might use this property to predict users choice in patterns. A theory is that humans would start their pattern in the same direction as their reading orientation. 

    {\bf $Q2.1$:} {\it What is your gender?}

    {\bf $Q2.2$:} {\it What is your age?}

    {\bf $Q2.3$:} {\it What is your current occupation?}

    {\bf $Q2.4$:} {\it Do you recognize yourself in any of the listed statements?}

    {\bf $Q2.5$:} {\it What is you main reading orientation?}

    \subsubsection*{Information not included in the questionnaire}
  
  \clearpage
  \subsection{Layout and Structure}\label{sec:layout}

    It how for now been discussed the background information and the demographics of the participants and the most essential missing is the respondents choice in patterns. Before the questionnaire starts the information will give a brief information of the use of data and privacy concerns. Collecting passwords is not a easy task, and will might look frightening to some respondents. After the information, the respondent will be asked to make a pattern that satisfies the rules of Android Unlock Patterns. The respondents will be asked to make a pattern they have used or would use on their smart phone, as well as making it remembered because they needs to retype the password in the end of the questionnaire. In this way we can reduce the risk of people making a too complex password that they probably would not use. When asking a person to make a ``complex password'', people tend to overcompensate and choose rather complicated passwords. This behavior is well known in psychology as ``priming''. By overcompensating and choosing rater more complicated patterns than probably would appear ``in the wild'' would introduce bias in the data because the overcompensated password is not representative. 

    After the first entry of the chosen pattern, there will be collected background and demographics information about the respondents (described in 4.3.2). This information with collect information that further will be used in the analysis of the collected patterns. After finishing the information, the respondent is asked to retype the password and is then finished.     

    \subsubsection*{Introduction}
    This introduction is giving the respondents information about the research. It is important to describe how the collected will be used in the research, as well as ensure privacy. There should not be possible to track the data back to a person and there will not be logged any additional data beside the answers that the respondent is answering. 

    \subsubsection*{Pattern training}
    If the participant are not familiar with the Lock Pattern, it can cause bias in the data. It will therefore be a opportunity for the respondents to practice before entering their chosen password. This will be formed like a ``tutorial'', and the selected patterns in the training mode will not be included in the data set. 
    Experienced respondents can choose to skip the training mode if there is no need for training. There will also be a description of the rules. 

    \subsubsection*{Background information}

    \subsubsection*{Selection of Pattens}
    In this part, there will be collected 3 different patterns. 
        \begin{enumerate}
            \item With what pattern would you choose to protect a shopping account?
            \item What pattern would you choose to protect your mobile phone?
            \item What pattern would you choose to protect your banking account?
        \end{enumerate}

    \subsubsection*{Demographic information} 


    \todo{Beskrive hvordan jeg vil illustrere svaralternativer med bilde.}
    \todo{Lage wireframes}

  \subsection{Wireframes}

  \subsection{Technical Design}\label{sec:technical}
    % When når jeg? SKal jeg nå alle på forskjellige OS? Begrenser designet mitt i noen grad?
    % Hvilke teknologier vil jeg bruke?

  \subsection{Pre-test and Pilot}\label{sec:pretest}
    When the questionnaire is ready, I need to perform a questionnaire to evaluate several aspects. First, I need to figure out if the respondents understand the questions stated in the questionnaire. This can cause a lot of bias in the data if the questions asked are ambiguous. Second, the time needed for completion of the questionnaire can not be too long. If the questionnaire takes too long to complete, there is less likely that people want to spend their time completing the questionnaire. It is stated that I need a sample size of 1000 for getting representative results for the analysis. At the same time, there is needed a lot of data to be able to make see patterns in the data. Therefore, it needs to be a balance between questions and data collected, and time of completion. 


    There is a need do some quality testing before distributing the questionnaire over the Internet. A pen and paper test will be tested to check the wording of the questions as well as getting feedback of the amount of information that is asked for. It is also interesting to ask questions about the ethical aspects of the data collection. If some of the test persons feels that their privacy is leaked by answering the questionnaire, the questions might needs to be evaluated for the final questionnaire. 
    This test will also be conducted in the following spring when the system for data collection is up and running. 

    \todo{GJENNOMFØRE EN UNDERSØKELSE PÅ SPØRSMÅLENE}
    %Forstår de hva som skal svares på? Ville de svart på denne undersøkelsen? Er etiske aspekter brutt? Vil det ta for lang tid?

  \subsection{Validity and Reliability}\label{sec:validityandreliability}

    {\bf Content validity}

    {\bf Construct validity}

    {\bf Reliability}




% Bør si noe om vanskeligheten med å samle inn password og forskjellen på å samle inn lekasje of faktiske passord.
% Hva bruker man mobilen til? Bank, facebook, mail, jobb, etc? Kan man se en sammenheng mellom passord og bruk av mobil. 
% Lage passord for forskjellige risikokategorier.

