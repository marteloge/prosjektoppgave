\section{Data Collection}

  \subsection{Form of Administration}

    This questionaire is a``Self-administered'', meaning that voulanterd respondants completes the questionaire without me being present. This choise is made due time saved, as well as getting enough data. A ``researcher-administrated'' questionaire would simple be too time consuming and there is a chance that it would not reach the required amount of collected data needed for the analysis. When sending the same questionaire over the internet, it eliminates the risk of introdusing bias by asking the respondants the questions in a different manner, and bias introduced by tone and body language. When collecting passwords the questionaire needs to be untrackable back to the respondants due to ethical considerations. A questionaire over the internet would make it easier for respondants to share patterns, but it also introduce the problem with the control of the responses and its realiability. The questionaire needs to be carfully designed in order to reduce possible biases. To get the data in a standarized format, the questionnaire will include closed-question and will therefore generate quantitative data. 

  \subsection{Question Content and Responses}

    \subsubsection*{Background information}
    To be able to analyse the data, it is a need to know something about the respondants background information that can either be useful in the analysis or be factors that may introduce bias in the data. For example, a person with no experience with the Andorid could may introduce bias because the user would not be familiar with its use and functionality. There is added question about the type of smartphone that the respondant is in currently in position of. The reason for asking this question is to be able to  check if the device used to answer the questionnaire provides the pattern lock security mechanism. In the analysis of human properties, there was selcted factors that may would inpact the choice of patterns. One of the properties that is selected is the hand size as well as the mobile size. These two properties would be interesting to analyse together. My own experiences shows that it its hard to reach certain areas of the screen with a small hand and a big mobile screen. \\

    \begin{table}[H]
      \begin{tabular}{| p{1cm} | p{10cm} |}
          \hline
          {\bf \#} & {\bf Formulation of Questions} \\ \hline
          1.1 & What kind of smartphone do you currently own? \\ \hline
          1.2 & What kind of locking mechanism do you currently use on your smartphone? \\ \hline
          1.3 & Have you ever used the Android Unlock Patten? \\ \hline
          1.4 & How would you categorize the screen size on your current smartphone? \\ \hline
          1.5 & Do you normally use a locking mechanism on your phone? \\ \hline
          1.6 & Are you left or right handed? \\ \hline
          1.7 & Which hand do you usually use for interactiong with yout mobile phone? \\ \hline
          1.8 & How would you categorize the size of your hand? \\ \hline
      \end{tabular}
      \caption{Questions - Background Information}
    \end{table}

    \subsubsection*{Demographic information}
    Demographic information can be used to group respondands in different subgroups to see if there is different choices in patterns in the subgroups. It it also valuable information to collect to see the diversity of the respondants. Some of the interesting human properties selected is the reading orientation. By studying the way humans read we can might use this property to predict users choice in patterns. A theory is that humans would start their pattern in the same direction as their reading orientation. 

    \begin{table}[H]
      \begin{tabular}{| p{1cm} | p{10cm} |}
        \hline
        {\bf \#} & {\bf Formulation of Questions} \\ \hline
        2.1 & What is your gender? \\ \hline
        2.2 & What is your age? \\ \hline
        2.3 & Which of the following alternatives describes your current occupation? \\ \hline
        2.4 & Which of the following alternatives describes your current profession? \\ \hline
        2.5 & {\color{red} What is your native language?} \\ \hline 
        2.6 & What is you main reading orientation? \\ \hline
      \end{tabular}
      \caption{Questions - Demographic Information}
    \end{table}

    \todo{Er native language overfløding når jeg direkte spør om ``reading orientation''?}

    %Age: 18 or younger,19 - 25, 26 - 30, 31 - 40, 41 - 50, 51 - 60, 61 - 70, 71 or older
    % Occupation: Science and technology, Engineering & Engineering Technicians, Industrial and manufacturing, Entertainment & media, Education, Teaching & Training, Farming & Fishing, Sales & Marketing, Military & Armed Forces, Office Administration & Management, Production & Manufacturing, Social & Life Sciences, Engineering, science and Technology, IT security, Humanities, Healthcare and medicine, Business & Finance, other.
  
  \clearpage
  \subsection{Layout and Structure}

    It how for now been discussed the background information and the demographics of the partisipants and the most essential missing is the respondants choice in patterns. Before the questionnaire starts the information will give a brief information of the use of data and privacy concerns. Collecting passwords is not a easy task, and will might look frightning to some respondants. After the information, the respondant will be asked to make a pattern that satisfies the rules of Android Unlock Patterns. The respondants will be asked to make a pattern they have used or would use on their smartphone, as well as making it rememberable becuase they needs to retype the password in the end of the questionnaire. In this way we can reduce the risk of people making a too complex password that they probably would not use. When asking a person to make a ``complex password'', people tend to overcompensate and choose rather complictaed passwords. This behavior is well known in psychology as ``priming''. By overcompensating and choosing rater more complicated patterns than probably would appear ``in the wild'' would introduce bias in the data because the overcomensated password is not representative. 

    After the first entry of the chosen pattern, there will be collected background and demographics information about the respondants (described in 4.3.2). This information with collect information that further will be used in the analysis of the collected patterns. After finishing the information, the respondant is asked to retype the password and is then finished.     

    \subsubsection*{Introduction}
    \subsubsection*{Selection of Pattens}
        \begin{itemize}
            \item  Select a pattern of at least 4 connected points that you currently are useing or would use as a password protection on your smartphone. 
        \end{itemize}
    \subsubsection*{Background information}
    \subsubsection*{Demographic information}
    \subsubsection*{Reentering of selected pattern}
        \begin{itemize}
            \item Select a pattern of at least 4 connected points that you currently are useing or would use as a password protection on your smartphone. 
            \item Did you select a pattern that you currently are using or have used?
                \begin{itemize}
                    \item Yes, I have used this pattern before
                    \item No, I have never used this pattern before
                    \item No comment
                \end{itemize}
        \end{itemize}

    \todo{Beskrive hvordan jeg vil illustrere svaralternativer med bilde.}

  \subsection{Pre-test and Pilot}
    \todo{Burde jeg gjennomføre dette?}

  \subsection{Validity and Reliability}




% Bør si noe om vanskeligheten med å samle inn password og forskjellen på å samle inn lekasje of faktiske passord.
% Hva bruker man mobilen til? Bank, facebook, mail, jobb, etc? Kan man se en sammenheng mellom passord og bruk av mobil. 
% Lage passord for forskjellige risikokategorier.

