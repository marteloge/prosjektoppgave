\chapter{Proposed Research Design}
  \label{chapter:researchDesign}

  In this chapter, we will continue with the results from the literature review by looking at a proposal for a research design. The content in this chapter will be a preparatory work for my master thesis that will be conducted in the following spring.

  Section~\ref{sec:hypothesis} is the proposed hypothesis to be used in my master thesis and is a result of the discussion of the literature review in Section~\ref{sec:resultsLiteratureStudy}. Section~\ref{sec:researchstrategy} is a discussion of the possible research strategies to adapt when conducting the research. The research strategy includes a review of the data that looks interesting to look further into based on the proposed hypothesis. Other aspects of the research design like sampling frame, sampling technique, sample size, and response rate is also discussed to provide a complete research design. When selecting a research strategy we need to choose the corresponding data collection method the selected data collection is determined in Section~\ref{sec:researchstrategy} and is further discussed in Section~\ref{sec:datacollection}. When looking at the data collection, this section will further describe the administrative part of the data collection, the questions to be asked, as well as a proposed prototype for collecting the data. The prototype is illustrated in Section~\ref{sec:wireframes}. Along with the wireframes, there is conducted a usability test. Description of the tests, the results, and suggested improvements is described. The prototype will be implemented in the next semester and will not be a deliverable for this project thesis. Section~\ref{sec:ethical} will be an evaluation of the ethical aspects of the research. Unethical research should not be accepted, and the research that is designed must be evaluated and argued to be within ethical guidelines.


  
   

 

  

  %   \subsection{Pattern Lock Functionality}
      
  %     \todo{Beskrive hvordan jeg forholder meg til Funksjonalitet som Android bruker}

  %     When the user first starts using the phone, they are prompted with the choise of using a locking mechanismn on the phone. The functionality of the Android Unlock Pattern are as follow: 
  %       \begin{enumerate}
  %         \item At least four points must be chosen,
  %         \item You cannot visit the same node twice.
  %         \item Only straight lines are allowed, and
  %         \item One cannot jump over point not visited before
  %       \end{enumerate}

    
  %   \subsection{Success Criterias}

  %       \todo{Diskutere hvilke faktorer som er viktige å være obs på i datainnsamlingen}


  % \section{Data Analysis}

  %   \todo{Bestemme hvilken type dataanalyse jeg vil ha. Dette er en følge av hvilke valg jeg har tatt i strategi. Quantitativ eller Qualitativ dataanalyse?}





