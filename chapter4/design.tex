\chapter{Research Design}
  
  In the litterature on graphical password it was found research with different focus. There have been published a lot of research that measures the usability of graphical passwords, but it still remains to look at the security of graphical passwords. However, there are not many researchers that have looked closer to the memorable password space and security of graphical password. As stated, mobile devices is a suited platform for graphical passwords, but there remains a lot of work here. The Android Unlock patterns is one of the graphical password schemes that are rapidly in use. As far as this study is familiar with, there is only one reaseach group that have published a large scale user study on the Android Unlock Patterns \cite{Ullenbeck}. One of the main challenges of working on graphical passwords is to be able to collect realiable user-chosen passwords. In contrast to text-based passwords were you can find lists of leaked passwords, this is unlikely to find for Android Unlock Patterns since they are stored in a centralized database. Passwords is something users dont give away easily, so how can we analyse user choice of password?  This study will design an experiment that will be conducted in the following spring. The experiment will collect data from users choices of patterns with and support a analysis with a new dimension of data. The goal is to be able to predict users choice of patterns based on a set of properties that are cloesly releted to the user making the pattern. 

 

  

  %   \subsection{Pattern Lock Functionality}
      
  %     \todo{Beskrive hvordan jeg forholder meg til Funksjonalitet som Android bruker}

  %     When the user first starts using the phone, they are prompted with the choise of using a locking mechanismn on the phone. The functionality of the Android Unlock Pattern are as follow: 
  %       \begin{enumerate}
  %         \item At least four points must be chosen,
  %         \item You cannot visit the same node twice.
  %         \item Only straight lines are allowed, and
  %         \item One cannot jump over point not visited before
  %       \end{enumerate}

    
  %   \subsection{Success Criterias}

  %       \todo{Diskutere hvilke faktorer som er viktige å være obs på i datainnsamlingen}


  % \section{Data Analysis}

  %   \todo{Bestemme hvilken type dataanalyse jeg vil ha. Dette er en følge av hvilke valg jeg har tatt i strategi. Quantitativ eller Qualitativ dataanalyse?}





