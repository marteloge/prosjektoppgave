\section{Research Questions and Hypothesis}
    
  {\bf $H_{0}$: Human properties have no influence in users choice of graphical passwords} 

  {\bf $H_{1}$: Users choice of graphical passwords are influenced by the human properties of the user}

\section{Research Strategy}

  \subsection{Selection of Research Strategy and Data Collection}

    In this thesis, the selected research strategy is a survey. To be able to answer the hypothesis there is a need for obtain data from a large group of people in a standardized and systematic way. This will provide a wide and inclusive coverage of people so the results from the data collection are likely to be representative for a wider population. 

    When selecting survey as a research strategy, a questionnaire would be a good fit for a quantitative data collection. The data properties that is selected are indicating that data needs to be collected ``world wide'', and cannot be collected face-to-face due to the lack of time and resources. For analyzing patterns in the data, a big amount of data needs to be collected. The questionnaire will be distributed over the Internet and will be helpful in the data collection process to get a sampling size that is large enough. Data collection with pen and paper are too time consuming and would not be manageable with the time available for data collection and analysis in the spring. When analyzing data, it is also necessary to have the data in a standardized format. A questionnaire is supporting a standardized format, and with a online tool it will be easy to extract the data in a standardized format for the analysis. This will provide the possibility to use automatically analysis tools without to much use of time with manually work.

    A limitations of the chosen approach is that it does not support control of the participants because the questionnaire is distributed over web, and users are not handpicked. Since the data collection is distributed over web, there is not possibility for me to judge the accuracy and honesty of peoples responses. Despite the limitations, this is chosen due to the amount of data needed for the analysis, as well as the lack of time for choosing other approaches like interviews and other observation techniques. 

    The detailed design of the questionnaire will be described in section 4.3.

  \subsection{Data Requirements}
  
  It is important to analyze what kind of human properties that may impact users choice of graphical passwords on mobile devices. We must carefully consider which human properties we want to collect in order to get a concise collection of data that can be analyzed, and further give us answer to our hypothesis. This analysis will be a review of human properties that this study find relevant to the hypothesis. As a result of this analysis, a narrowed selection of the listed human properties will further be included in the analysis.  

  \begin{itemize}
    \item {\bf Age:} A group of people within a certain group of age may have different risk awareness. A person with a age between 30-40 and a person with a age below 20 may have different concerns with security. A person in the age between 30-40 may use their phone to perform task with a high security risks like making and job related tasks, while a person with a age below 20 may not have the same security awareness because of the different use of mobile devices.
    \item {\bf Gender:} Psychological studies have reported that males are more likely to take risks than females \cite{Byrnes}. When looking at peoples choice in patterns can probability be analyzed based on gender. In the literature review there was no reported results found on genders risk awareness in information security, nor peoples choice of patterns based on gender. By analyzing peoples choice of patterns based on users gender might give interesting results. 
    \item {\bf Nationality:} 
    \item {\bf Language preference:}
    \item{\bf Occupation:}
    \item {\bf Profession:} The profession of a person may say something about a persons knowledge and background. When looking at profession, a person with a profession in IT may be more certain about the security aspects than people in other professions. 
    \item {\bf Left- or right handed:} An interesting property of humans is the fact that people write with either left or right hand (and sometimes both). This property can influence the way a person are holding the phone and may impact the way that a person is making a pattern. In the literature review it was not found any studies that reported any results of people choices in patterns based on the hand used. Published research \cite{Uellenbeck} found that over 40\% of interest in a study started their Android pattern by starting in the top left corner, but did not record the hand used when making the pattern. My hypothesis is that a left handed person may be using the left hand while interacting with the phone, making the probability for starting in the right upper corner bigger. This have never been tested before and need further research before making a statement. 
    \item {\bf Reading/writing orientation:} In different cultures, there is a difference in the reading and writing direction. Cultures from Europe and America is normally writing and reading horizontally from left to right, but there are other cultures that do otherwise. Traditionally, Chinese, Japanese, and Korean are writing text vertically in columns from top to bottom, from right to left. Another writing orientation is horizontal from right to left that are used in Arabic cultures. Today, the vertical orientation from top to bottom is often in a horizontal way due to the influence of English and the increased computerized typesetting, but both ways are still in use. There are research that have reported that the writing orientation are affecting the visual attention and memory \cite{Chan}. They found that the reading orientation affected the way a person would memorize objects. They reported that English and Chinese speakers tended to remember an image that appeared in the top, left-hand side of the screen and the Taiwanese speakers tended to remember the images in the upper right-hand side of the screen. The interesting aspect of this reading and writing orientation is to see if people from different cultures are choosing different patterns due to their writing orientation.
    \item {\bf Hand size:} Smartphones today tends to get bigger and bigger in size. An interesting analysis could be done by looking at a users choice of patterns based on size of their hands and size of mobile phone where a patterns is made. By looking at a situation were a person with a small hand and a big mobile device, it may be hard to reach certain areas on the screen by holding a device in one hand, and therefore impact the choice of pattern. 
  \end{itemize}

  \begin{figure}[H]
    \centering
    \subfigure{
      \includegraphics[scale=0.25]{pics/leftright.png}
    }
    \subfigure{
      \includegraphics[scale=0.25]{pics/rightleft.png}
    }
    \subfigure{
      \includegraphics[scale=0.27]{pics/topbottom.png}
    }
    \caption{Writing orientatio: 1) Horizontally Left-to right, 2) Horizontally Right-to-left 3) Vertically Right-to-left}
  \end{figure}

  {\color{red} Jeg vil velge å se videre på: "reading/writing orientation", "hand size", "Left- or right-handed" som hovedpunktene. Tror uansett det er lurt å samle inn annen demografisk informasjon.} 

  \todo{Vil det være lurt å spørre om innhold på mobil og bruksområde? Vil det ha noe utslag på feks styrke på mønster?}

  \subsection{Sampling Frame and Technique}

    \todo{Er det her jeg skal nevne hvem jeg ønsker å samle inn svar fra (sampling frame)?}
    
    The survey type is categorized as ``non-probability sampling'' and the chosen sample technique is ``self-selection strategy''. This is chosen due to time and cost estimates, and the lack of control of participants. A Purposive sampling technique would maybe provide a more uniformly collection of people, but it is hard to control the participants when the questionnaire is distributed on the web. The self-selection strategy will collect data from any respondents, and will be helpful when there is big population with potential respondents. The self-selection is a useful technique when we are not able to directly contact the potential respondents. When people select themselves for research, it might indicate that they have a strong feeling on the subject, or because they think it will bring them personal benefit or approval. 

  \subsection{Response Rate and Non-responses}

    When sending out the questionnaire I have no control over how many people will participate because of the distribution of the questionnaire over the Internet. To be able to reach the amount of data need I need to look for a strategy that may increase the number of responses. If I suspect that certain types of people in my sample will be less willing to respond, I could deliberately include more of that type in my sample so I can assure that I receive the number of respondents that I need. Maybe go face-to-face if a special group of people may be less willing to participate. 

    In this table there will be described different subgroups of people that I want to get data from, but may be hard to reach with respect of different factors. There will be description of a strategy for reaching the subgroups that I predict to have less responses from:

    \begin{tabular}{| p{4cm} | p{7cm} |}
      \hline
      {\bf Subgroup} & {\bf Sampling strategy} \\ \hline
      People with age of 50 or higher & \\ \hline
      Subgroups with a different field of interest than IT & \\ \hline
      Subgroups with a reading orientation from right-to-left & \\ \hline
      Subgroups living in a different country than Norway & I need to get in touch with persons from different nationalities to get diversity in the data. I will contact the International Section at my university to ask them to distribute my questionnaire to the exchange students at my university. I am also having a trip to Minnesota in USA in the following spring, and I will use my time there recruiting people to respond to my questionnaire. During this semester I have been talking to researchers in other countries because of their interest in my research. Hopefully I can get help to distribute the questionnaire within their network. \\ \hline
      Subgroups that are left-handed & There is a significant higher percentage of people that are right-handed, meaning that there is a possibility for recruiting more people that are right-handed. Selecting a strategy for this is hard because there is no forum for left-handed people. If there are few respondents that are left-handed I need to directly look for people that are left-handed at my school.\\ \hline
    \end{tabular}

    When analyzing the collected data in the following spring, I need to find characteristics of people that have not answered to make a discussion of why they didn't (maybe they are not representative), or whether their non-response is meaningful in its own right, or whether their lack of responses has resulted bias in my final sample. This part is out of scope for this thesis, but need to be considered in my thesis next semester.
  
  \subsection{Sample Size}
    I need to decide how big I want my final sample to be, taking into account my best estimate of the likely non-response rate of participants. 
    Good rule of thumb is to have at least 30. For a target population of 1 million or more, I would probably need 1000. 

    The Internet offers researchers the possibility of accessing many people across the world cheaply and quickly. Not everyone have Internet access, and that need to be considered in the design. Maybe they are not a targeted group since this is a mobile focus.

