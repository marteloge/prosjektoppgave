\section*{Abstract}
  
  Since the first proposal for a graphical password around 1999, a variety number of graphical password schemes have been proposed. The proposed graphical password schemes were motivated by the promise of improved password memorability and thus usability, while at the same time trying increase the security.  Psychology studies have recognized that the human brain have a superior memory for recognizing and recalling visual information rather than recognizing and recalling verbal or textual information. Graphical passwords seems as a good replacement of text-based authentication. 

  The motivation for this thesis started by observing the shortcomings with text-based authentication, where people tend to obtain bad habits because of the difficulty of remembering the textual information. People therefore tend to create easily guessed passwords. Password sharing and password reuse are also some of the know habits that people obtain by using text-based passwords. When looking at mobile devices, text-based passwords are not easily typed on a mobile screen. Graphical passwords do not only seem as a good replacement of text-based passwords, but also look like a great authentication method for mobile devices because of the easily interaction with graphical elements on a small touch screen. 

  Android Unlock Patterns is one of the graphical password schemes with an commercial success on mobile devices. The only large-scale study that have conducted quantified the security of peoples choice in patterns. This study aims to take the analysis of people's choice in Android Unlock Patterns a step further by including the human properties that may impact the user's choice in graphical passwords. I believe that graphical passwords are more than just pictures and graphical objects.

  In this thesis there is conducted a literature review of graphical passwords. There is also created a proposal for a research design for further continuation of this work. The research design contains a research strategy, as well as a prototype for data collection. 

  