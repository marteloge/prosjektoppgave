\chapter{Introduction}

\section{Background}
    
  In today's society we're addicted to our mobile devices in our every day life. 
  Mobile devices are not just a communication tool for calling and texting, 
  but also an important tool for every day tasks like doing our work, reading mail, pay our bills and keeping up 
  with our social life. Our whole life is contained in one device!
  When such a small device is so imortant, it makes it vurnerable. What could happend if you lost it?
  Since our mobile devices gets more and more important in our daily life, how do we secure it?

  There are many different security tips for mobile devices, but the most imortant of them is locking your device
  with a password. There are many different password schemes, but two of the most popular ones are alphanumeric 
  passwords, such as PIN's, and graphical passwords like the Android Unlock password. 

  The interest in graphical passwords started by the assumption that pictures are easier to remember and more 
  secure than words. Google's Android platform released the functionality for Unlock Pattern in 2008 to draw a 
  pattern in a 3x3 grid to unlock your phone. Since its relese there have been a lot of discussion 
  of its empirical password space, but few researchers have done a scientific reseach on its security. 

  In 2013 a research group conducted the first large-scale user study on Android Unlock Patterns \cite{Uellenbeck}. 
  The outcome of the research was a analysis of 2900 collected Android Unlock Patterns.



\section{Motivation}

    % - Fra empiriske boka kap 2: 
    %   * Test or disprove a theory
    %   * To come up with a better way
    
    % - Paper til Markus.
    % - Fortsette på studiet til Markus. 
    % - Per Thorsheim sine studier på PIN koder
    % - I dagens samfunn inneholder mobile enheter mer og mer sensitiv informasjon, 
    %   men våre kunnskaper om autentisering og sikkerhet henger ikke med i samme fart. 
    % - Se sammengengen mellom personer og deres valg av mønstre
    % - Se sammenheng mellom psykologi og teknologi/sikkerhet

    % Quote: “Somewhere, something incredible is waiting to be known.” 
    % ― Carl Sagan

    % “If you have built castles in the air, your work need not be lost; that is where they should be. Now put the foundations under them.” 
    %   ― Henry David Thoreau, Walden


\section{Research Questions}

  \begin{enumerate}

      % \item To what extent can graphical elements like colors, shapes, and objects infuence the end-users choice of passwords?
      % \item How is features describing the end-user picked, and how do the features relate to end-user's choice of passwords?
      % \item Kan grafiske elementer påvirke brukerens valg av passord?
      % \item Hvilke kjennetegn ved en person kan gi utslag på valg av passord?
      % \item Hvordan skal innsamling av passord skjer for å ivareta datasamlingens pålitelighet?

      \item To what extent can we relate existing research on users choice of alphanumeric passwords to users choice of Android unlock patterns? 
      \item How should passwords from end-users be collected in order to preserve the reliability of the data? 
      \item In order to analyse collected Android unlock patterns, how much data is needed to be collected, and how should the diversity of the data look like?
      \item What kind of data should be included in the data model?
      \item How should the data model be designed in order to cover relevant data for the analysis of the collected data?  

  \end{enumerate}

\section{Goals}

  \begin{itemize}
        \item Designing the system for collecting data
        \item Prepare for data collection (how should I do this?)
        \item Get an overview of reserach done on Android Unlock pattern, and what research questions remains undone.
  \end{itemize}

\section{Expected Deliverables}

\section{Research Method}
 
    This chapter will include a description of the research method used for this projekt.



